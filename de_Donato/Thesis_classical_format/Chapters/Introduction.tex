\chapter{Introduction}
\label{ch:introduction}

It is widely agreed that, among the many elements that make a game enjoyable, \textit{level design} is one of the most important. \textit{Level design} refers to the process of designing and creating the worlds and environments where the player interacts and plays. Despite its importance, \textit{level design} is a complex and time-consuming task, which requires extensive knowledge and experience. 

In order to create engaging experiences for players games may require a lot of content to be designed and created, which can be a time-consuming and expensive process. In order to create games with more content and that offer more replayability, the industry has seen the rise of \textit{Procedural Content Generation} (PCG) techniques, which refer to the class of algorithms and methods used to generate content automatically. Critically acclaimed games that range from indies, such as \textit{Minecraft}, to AAA titles, such as \textit{No Man's Sky}, employ PCG techniques to various degrees to offer ever-changing gameplay. 

One relevant question in the context of PCG is how to generate content that is not only of high quality, but also novel and diverse. \textit{Quality Diversity} optimization refers to a relatively new class of algorithms that, taking inspiration from the process of biological evolution, aims to answer this very question by generating a diverse set of high-quality solutions, instead of a single one. In the literature, \textit{Quality Diversity} algorithms have been used to generate content for different genres, such as platformers and puzzle games.

\textit{First-Person Shooters} (FPS), especially multiplayer competitive ones, are among the most popular games in the industry. Despite this, the topic of level design for FPS has seen only a few scientific works which have tried to define patterns and techniques, but ultimately still lacks specific standards. Moreover, PCG has yet to be used in a critically acclaimed competitive FPS, despite having shown great potential through its application in other genres. Contrary to the industry, scientific studies have taken an increasing interest in the use of PCG in FPS, with \textit{evolutionary algorithms} being used to generate maps. However, despite being particularly suited for PCG, given their focus on both diversity and quality, \textit{Quality Diversity} algorithms have yet to be applied to FPS games.

Our aim with this thesis is that of filling this gap by exploring the use of a Quality Diversity algorithm, \textit{MAP-Elites with Sliding Boundaries} (MESB) \cite{fontaine_mapping_2019}, to the generation of maps for multiplayer competitive FPS games. We will define ways to represent maps and extract metrics that describe each map, study the correlation between these metrics and use them to search for diverse and high-quality maps using \textit{MESB}. We will compare the performance of different representations and discuss the results obtained.

\section{Contents of the thesis}
We structure our thesis as follows:

In the second chapter we will explore the state of the art regarding the main concepts of interests in our work, which include \textit{level design} for FPS games, the \textit{balancing} of competitive multiplayer games and the use of Procedural Content Generation (PCG) in games. We will particularly examine \textit{search-based} PCG, which will allow us to understand how PCG has been applied to the generation of FPS maps in the literature. Finally, we will introduce the concept of \textit{Quality Diversity} along with how it has been used for the generation of content in games.

In the third chapter we will present the frameworks we have used to conduct our research: \textit{Project Arena}, a research-oriented framework for FPS games developed in Unity by \citet{ballabio_online_2018}, and \textit{pyribs} \cite{tjanaka_pyribs_2023}, a Python library for Quality Diversity algorithms. We will explain the reason for our choices along with a description of their capabilities and structure. 

In the fourth chapter we will first describe the genome representations used to describe and evolve maps. We will present All-Black \cite{cardamone_evolving_2011} and Grid-Graph \cite{bari_evolutionary-based_2023}, which have already been used to evolve maps with evolutionary algorithms, and introduce our own representations, which we called \textit{Point-Line} and \textit{SMT-Genome}. Then, we will discuss the features we have extracted from each map, categorizing them in \textit{emergent}, which are directly extracted from simulated gameplay, and \textit{topological}, which are extracted from the map's structure.

In the fifth chapter we will present the experiments we have conducted. First, we are going to make sure that the emergent features chosen are not noisy, making them suitable to describe maps. Then we will analyze the correlation between different features to establish patterns and groups of features. Following, we will select relevant features to illuminate leveraging the result of running the \textit{t-SNE} on different high-dimensional spaces that describe the maps. Finally, we establish three \textit{behavioral characterizations} and run the \textit{MAP-Elites with Sliding Boundaries} algorithm, analyze the results and compare the performance of the different representations.

Finally, we sum up the work done, discuss the results, argue possible shortcomings of our work and propose prospects for future work.
