\chapter{Conclusions and future developments}
\label{ch:conclusions}

This thesis' purpose was to explore the feature space of competitive First-Person Shooter maps by defining relevant measures to describe maps which could then be used in a Quality Diversity algorithm for the procedural content generation of new maps. We were interested in the use of different representations and their comparison in terms of algorithm performance and map design quality.

We achieved our goal through the study of relevant literature in the field of level design applied to competitive First-Person shooters, using this knowledge to try and define new features that could be used to effectively describe maps both topologically and in terms of gameplay emerging from actual play of the map. We have been able to define novel and interesting topological features by using room segmentation strategies to describe the map as a graph and by using approximate grid-based methods to extract information about the maps' visibility. We have instead extracted emergent gameplay features by analyzing the gameplay data of matches simulated using bots in \textit{Project Arena}, a research oriented First-Person Shooter framework developed by \citet{ballabio_online_2018}, later extended to support bots by \citet{bari_evolutionary-based_2023}.

Maps were represented using genome representations relevant in the literature (\textit{All-Black}, introduced by \citet{cardamone_evolving_2011} and \textit{Grid-Graph}, introduced by \citet{bari_evolutionary-based_2023}) and using novel representations defined by us (\textit{Point-Line}, inspired by the work of \citet{olsted_interactive_2015} and \textit{SMT-Genome}, inspired by the work of \citet{whitehead_spatial_2020})

We have studied the noisiness of the features that we have defined and their correlations with one another in order to visualize correlation patterns between features and to define interesting couples of strongly negatively correlated features to be used to define the \textit{behavioral characterization} to be used in the Quality Diversity algorithm. We then used the \textit{t-SNE} dimension reduction algorithm to visualize the high dimensional spaces of the features, of the images and of the graphs of the maps in a 2D space, in order to understand interesting patterns in the data and to visualize the relationship between a map's image or graph and its features. We leverage the results to define which features have an impact on the map's graph topology, while no relevant correlation was found between the features and the map's image.

Thanks to these preliminary analyses, we have been able to define three relevant \textit{behavioral characterizations} to be used with the \textit{Map Elites with Sliding Boundaries} algorithm, a Quality Diversity algorithm, to evolve and generate new maps, focusing simultaneously on optimizing the balance (\textit{entropy}) of the maps and on exploring the feature space, which is the underlying goal of QD algorithms. Results have shown that, while all representations are able to produce maps that achieve balanced gameplay, not all representations explore the feature space fully nor produce quality map's designs. The \textit{Grid-Graph} representation has been shown to be the least effective at illuminating the feature space, while producing elites with high entropy but low quality and diversity of designs. \textit{All-Black} performed better in terms of diversity but generated noisy designs with many useless topological features. \textit{Point-Line} and \textit{SMT-Genome} have been shown to be the most effective at exploring the feature space, producing maps that are significantly cleaner than \textit{All-Black} while sporting a higher diversity of designs.

Our results suggest that choosing the correct representation is crucial to both the success of the illumination and to the quality of the produced maps in terms of actual gameplay and that the feature we have chosen have led to discovering diverse map designs that are balanced in terms of gameplay, as was our goal.

\section{Shortcomings and future developments}

While the results of this thesis are promising, there are still shortcomings that could be addressed in future work. 

The first possible criticism regards the use of bots to simulate gameplay. While this was the only viable option to evolve maps over simulations of hundreds of iterations, bots are inherently unable to fully replicate the behavior of human players, which are the ultimate target of any game. Future work could focus on mixing human evaluation in a co-creativity setting, where players or designers could interact with the algorithm to provide feedback and guide the search, as already done for dungeon designs by \citet{alvarez_interactive_2022}.

While optimizing entropy is a design objective that has been used in the literature multiple times, it is apparent from our results how entropy does not relate with good map design. Future work could focus on defining new objectives that are more closely related to the quality of the maps, while not overlooking the importance of balance in FPS maps. This, however, may lead to problems that are hard to define and measure, as the quality of a map is subjective and may vary from player to player.

Future work could try and define new representations that overcome the locality problems we discussed for all the representations presented in this thesis, except for \textit{Grid-Graph}, which specifically addressed this issue but failed to produce diverse maps as a result. New interesting metrics may also be defined by looking at spawn-points or by using objects, such as ammunition or health packs, or by switching to a team-based game mode and analyzing team dynamics.  

