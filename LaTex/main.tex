\documentclass[11pt,letterpaper,twoside,openrigth]{book}

\usepackage[ruled]{algorithm2e}
\usepackage{amsmath}
\usepackage{amssymb}
\usepackage{booktabs} 
\usepackage{caption}
\usepackage{emptypage}
\usepackage{fancyhdr}
\usepackage[hang,flushmargin]{footmisc}
\usepackage{graphicx}
\usepackage[utf8]{inputenc}
\usepackage{ragged2e}
\usepackage{setspace}
\usepackage{siunitx}
\usepackage{subcaption}
\usepackage{tabularx}
\usepackage{titlepic}
\usepackage[nottoc]{tocbibind}

% GENERAL SETUP %

% Customize header and footer.
\pagestyle{fancy}
\fancyhf{}
\fancyhead[LE]{\slshape\nouppercase{\textnormal{\leftmark}}}
\fancyhead[RO]{\slshape\nouppercase{\textnormal{\rightmark}}}
\fancyfoot{}
\fancyfoot[RO, LE]{\thepage}

% Force footnotes to stay on the same page they are defined.
% \interfootnotelinepenalty=10000

% Limit decimal to 2 digits.
\sisetup{
  round-mode = places,
  round-precision = 2,
}

% Limit the TOC to the subsubsections.
\setcounter{tocdepth}{3}

% Set the image folder path.
\graphicspath{ {images/} }

% Relax italics in the TOC.
\let\LaTeXStandardTableOfContents\tableofcontents
\renewcommand{\tableofcontents}{\begingroup\renewcommand{\itshape}{\relax}\LaTeXStandardTableOfContents\endgroup}

% Define footnote for algorithms.
\makeatletter
\newcommand{\algorithmfootnote}[2][\footnotesize]{
	\let\old@algocf@finish\@algocf@finish
	\def\@algocf@finish{\old@algocf@finish
	\leavevmode\rlap{\begin{minipage}{\linewidth}
	#1#2\end{minipage}}}
}

% Add ''and'' and ''or'' keywords to the algorithm environment.
\SetKw{And}{and}
\SetKw{Or}{or}

% Define centering column with automatic newline in table.
\newcolumntype{C}{>{\Centering\arraybackslash}X}

% PDF SETUP %

% Set the left and the right margin to the same width.
\setlength\oddsidemargin{\dimexpr(\paperwidth-\textwidth)/2 - 1in\relax}
\setlength\evensidemargin{\oddsidemargin}

% MACROS %

% Macro for italics.
\def\<#1>{\textit{#1}}
% Macro for bold.
\def\[#1]{\textbf{#1}}

% FRONT MATTER %

\begin{document}

\frontmatter

% TITLE PAGE

\begin{titlepage}
\centering

{\scshape\LARGE Politecnico di Milano \par}
{\Large Corso di Laurea Magistrale in Ingegneria Informatica \par}
{\Large Dipartimento di Elettronica, Informazione e Bioingegneria}

\vspace{1cm} \includegraphics[width=0.3\textwidth]{polimi}\par \vspace{1cm}

{\scshape\LARGE Development of a framework for user-based validation of FPS-oriented level design research}

\raggedright
\vfill\Large Supervisor: Professor Daniele LOIACONO

\vfill \begin{minipage}[t]{0.4\textwidth}\end{minipage} \hfill
\begin{minipage}[t]{0.59\textwidth} \begin{flushright}
\raggedright Final thesis by: \par Marco BALLABIO Matr. 857169
\end{flushright} \end{minipage}

\centering
\vfill\Large Academic year 2017/2018

\end{titlepage}

% ABSTRACTS AND THANKS %

\chapter{\textit{Thanks}}

\textit{I would like to thank Assistant Professor Daniele Loiacono, that supported me during the long six months that led to the completion of this work.}

\par \mbox{}

\textit{Thank also to my whole family and to my colleague and friend Luca, essential companion in this journey.}

\par \mbox{}

\textit{Finally, I would like to thank Politecnico di Milano itself, of which I can call myself a proud student.}

\par \mbox{}

\textit{\rightline{Marco Ballabio}}

\chapter{Abstract}

\<Level design> plays a key role in the development of a video game, since it allows to transform the \<game design> in the actual \<gameplay> that the final user is going to experience. Nevertheless, we are still far from a scientific approach to the subject, with a complete lack of a shared terminology and almost no experimental validation for the most used techniques. Even if the video game industry doesn't acknowledge this problem, in the last years the academic environments have shown an increasing interest towards this subject. \\
We analyzed the main breakthroughs made in \<level design> research applied to the genre of \<First Person Shooters>, devoting particular attention to the ones that try to assist the design process by employing \<Search-Based Procedural Content Generation> combined with \<evolutionary algorithms>. We noticed that in most cases researchers recur to \<validation> via \<artificial agents>, since human play-test session are too time-consuming and allow to collect only a limited amount of data. Unfortunately, this solution decreases the scientific soundness of the obtained results, since the behavior of an AI, no matter how advanced, is different form the one of a human player. To solve this problem, we developed an \<open-source> \<framework> capable of deploying online experiments to collect data from real users via a browser FPS game. \\
We also explored a novel approach to procedural generation of contents, developing a tool that uses Graph Theory to displace spawn-points and resources in a procedurally generated map. Finally, we used our framework to validate this tool by means of an online data-collection campaign. 

\chapter{Sintesi}

Il \<Level design> gioca un ruolo chiave nello sviluppo di un videogioco, dal momento che permette di trasformare il \<game design> nell'effettiva esperienza di \<gameplay> che verrà sperimentata dall'utente finale. Nonostante ciò, siamo ancora lontani da un approccio scientifico verso la materia, a casusa della completa mancanza di un vocabolario condiviso e della quasi totale assenza di validazione sperimentale per le tecniche più comuni. Anche se l'industria tende ad ignorare questo problema, negli ultimi anni gli ambienti accademici hanno mostrato un crescente interesse verso questo campo. \\
Abbiamo analizzato le principali scoperte fatte nel campo del \<level design> applicato al genere dei \<First Person Shooter>, riservando particolare attenzione ai casi dove si usa la \<generazione procedurali di contenuti> tramite \<algoritmi evolutivi> per assistere il processo di design. Abbiamo notato che nella maggior parte dei casi i ricercatori ricorrono alla \<validazione> tramite \<agenti artificiali>, dal momento che organizzare sessioni di test con giocatori umani richiede molto tempo e permette di raccogliere una quantità limitata di dati. Dal momento che il comportamento di un'IA, per quanto avanzata, è molto diverso da quello di un giocatore umano, questa soluzione va a diminuire la solidità scientifica dei risultati ottenuti. Per ovviare a questo problema, abbiamo sviluppato un \<framework> \<open-source> per la raccolta di dati online da utenti reali, tramite un gioco FPS usufruibile da browser. \\
Abbiamo anche tentato un nuovo approccio alla generazione procedurale di contenuti, sviluppando uno strumento che utilizza la Teoria dei Grafi per disporre punti di respawn e risorse di vario tipo in una mappa generata proceduralmente. Abbiamo infine usato il nostro framework per validare questo strumento tramite una raccolta di dati online.

\tableofcontents

\listoffigures

\listoftables

% MAIN MATTER %

\mainmatter

% CHAPTER 1 - INTRODUCTION AND MOTIVATION %

\chapter{Introduction}

Developing a video game is a really complex process, which involves a wide range of professional figures. Since video games are interactive products, one of the most important roles is the game designer, who defines the game design and the gameplay. 

% CHAPTER 2 - STATE OF THE ART %

\chapter{State of the art}

% INTRODUCTION %

In this chapter we analyze the current state of \<Level Design> and of its common practices, both in academic and in professional environments, with attention to the genre of \<First Person Shooters> (or \<FPS>).

\par

We then talk about \<Procedural Content Generation> (or \<PCG>), focusing on how it allows to enrich and ease the design process.

\par

Finally, we give an overview of the First Person Shooter genre, analyzing its features, history and evolution, devoting special attention to the games that lead to greater innovation in the field and to the ones that are used to perform academic research in this field.

% LEVEL DESIGN %

\section{Level Design Theory}

\<Level Design> is a game development discipline focused on the creation of video game levels.

\par

Today, the level designer is a well-defined and fundamental figure in the development of a game, but it was not always so. In the early days of the video game industry, it was a widespread practice to assign the development of levels to members of the team with other roles, usually programmers. Apart from the limited number of team members and budget, this was because there were no tools such \<level editors>\footnote{\label{levelEditorFootnote}A level editor is a software used to design levels, maps and virtual worlds for a video game. An individual involved with the creation of game levels is a level designer.}, that allowed the level designer to work on a level without being involved with code.

\par

The level designer has a really significant role in the development of a good game, since he is responsible for the creation of the world and for how the player interacts with it. The level designer takes an idea, which is the game design, and makes it tangible.
Despite the importance of this role, after all this years, it has not been established a common ground or a set of standards yet, instead, level design is often considered as a form of art, based on heuristics, observation, previous solutions and personal sensibility.

\par

In addition to game play, the game designer must consider the visual appearance of the level and the technological limitations of the \<game engine>\footnote{\label{gameEngineFootnote}A game engine is a software framework designed for the creation and development of video games.}, combining all this elements in a harmonic way.

\par

One of the core components of level design is the \[''level flow'']. For single player games it translates into the series of actions and movements that the player needs to perform to complete the level. A proficient level design practice is to guide the player in a transparent way, by directing his attention towards the path he needs to follow. This can be achieved in diverse ways. Power ups and items can be used as breadcrumbs to suggest the right direction in a one-way fashion, since they disappear once picked up. Lighting, illumination and distinctly colored objects are another common approach to this problem. A brilliant example of this is \<Mirror's Edge>\footnote{Digital Illusions CE, 2008.}, which uses a really clear color code, with red interactive objects in an otherwise white world, to guide the player through its fast-paced levels. There are also even more inventive solutions, like the dynamic flock of birds in \<Half Life 2>\footnote{Valve, 2004.}, used to catch the player attention or to warn him of incoming dangers\cite{GuidingThePlayersEye}. Finally, sounds and architectures are other elements that can be used to guide the player. In the academic environment, a lot of researchers have analyzed the effectiveness of this kind of solutions: Alotto\cite{HowLevelDesignersAffect} considers how architecture influences the decisions of the player, whereas Hoeg\cite{TheInvisibleHand} also considers the effect of sounds, objects and illumination, with the last being the focus of Brownmiller's\cite{InGameLigthing} work.

\par

In multiplayer games the level flow is defined by how the players interact with each other and with the environment. Because of this, the control of the level designer is less direct and is exercised almost exclusively by modeling the map. Considering FPS, the level flow changes depending on how much an area is attractive for a player. The more an area is easy to navigate or offers tactical advantage, such as cover, resources or high ground, the more players will be comfortable moving in it. This doesn't mean that all areas need to be designed like this, since zones with a ''bad'' flow but an attractive reward, such as a powerful weapon, force the player to evaluate risks and benefits, making the game play more engaging. The conformation of the map and the positioning of interesting resources are used to obtain what Güttler et al.\cite{Guttler:2003:SPL:963900.963915} define as \[''points of collisions''], i.e. zones of the map were the majority of the fights are bound to happen. Moving back to academic research, Güttler et al. have also noticed how aesthetic design loses importance in a multiplayer context. Other researches are instead focused on finding \[patterns] in the design of multiplayer maps: Larsen\cite{LevelDesignPatterns} analyzes three really different multiplayer games, \<Unreal Tournament 2004>\footnote{Epic Games, 2004.}, \<Day of Defeat: Source>\footnote{Valve, 2005.} and \<Battlefield 1942>\footnote{DICE, 2002.}, identifying shared patterns and measuring their effect on gameplay, suggesting some guidelines on how to use them, whereas Hullet and Whitehead identify some patterns for single player games\cite{Hullett:2010:DPF:1822348.1822359}, many of whom are compatible with a multiplayer setting, with Hullett also proving cause-effect relationships for some of this patterns by confronting hypnotized results with the ones observed on a sample of real players\cite{TheScienceOfLevelDesign}. Despite these experimental results contributing to a formalization of level design, we are still far from a structured scientific approach to the subject.

% PROCEDURAL CONTENT GENERATION %

\section{Procedural Content Generation}

\<Procedural Content Generation> refers to a family of algorithms used to create data and content in an automatic fashion. In game development it is commonly used to generate weapons, objects, maps and levels, but it is also employed for producing textures, models, animations, music and dialogues.

\par

The first popular game to use this technique was \<Rogue>\footnote{Michael Toy, Glenn Wichman, 1980.}, an ASCII dungeon exploration game released in 1980, where the rooms, hallways, monsters, and treasures the player was going to find were generated in a pseudo-random fashion at each playthrough. Besides providing a huge replay value to a game, PCG allowed to overcome the strict memory limitations of the early computers. Many games used pseudo-random generators with predefined \<seed values> to create very large game worlds that appeared to be premade. For instance, the space exploration and trading game \<Elite>\footnote{David Braben, Ian Bell, 1984.} contained only eight galaxies, each one with 256 solar systems, of the possible 282 trillion the code was able to generate, since the publisher was afraid that such an high number could cause disbelief in the players. Another example is the open world action role-playing game \<The Elder Scrolls II: Daggerfall>\footnote{Bethesda Softworks, 1996.}, which game world has the same size as Great Britain. 

\par

As computer hardware advanced and CDs become more and more capacious, procedural generation of game worlds was generally put aside, since it could not compete with the level of detail that hand-crafted worlds were able to achieve.

\par

However, in the last years, with the players' expectations and the production value of video games constantly increasing, procedural generation made a comeback as a way to automate the development process and reduce costs. Many \<middleware> tools, as \<SpeedTree>\footnote{IDV, Inc.} and \<World Machine>\footnote{World Machine Software, LLC.}, are used to produce content, like terrain and natural or artificial environments.

\par

Many modern \<AAA>\footnote{Video games produced and distributed by a major publisher, typically having high development and marketing budgets.} games use procedural generation: in \<Borderlands>\footnote{Gearbox Software, 2009.} a procedural algorithm is responsible for the generation of guns and other pieces of equipment, with over a million unique combinations; in \<Left 4 Dead>\footnote{Valve, 2008.} an artificial intelligence is used to constantly make the players feel under threat, by dynamically changing the music, spawning waves of enemies and changing the accessible paths of the level; in \<Spore>\footnote{Maxis, 2008.} \<procedural animation> is employed to determine how the creatures created by the player move.

\par

Nowadays, PCG is widely used by \<independent> developers, that, lacking the high budgets of AAA games, try to obtain engaging and unusual gameplay using unconventional means. The most famous example is \<Minecraft>\footnote{\label{ }Mojang, 2011.}, a sandbox survival game which worlds, composed exclusively by cubes, are generated automatically. Currently, the most extreme form of procedural generation is the one found in \<No Man's Sky>\footnote{Hello Games, 2016.}, a space exploration game, where space stations, star-ships, planets, trees, resources, buildings, animals, weapons and even missions are generated procedurally. Following in the footstep of their forefather, many roguelike games still use PCG, like \<The Binding of Isaac>\footnote{Edmund McMillen, 2011.}.

\par

All the algorithms used by these games and middleware are design to be as fast as possible, since they need to generate the content in real time. In the last years researchers have nevertheless tried to explore new paradigms, creating more complex procedural generation techniques, that allow for a tighter control on the output. Being one of the problems of PCG the lack of an assured minimum quality on the produced content, the academic environment has focused not only on more advanced generation algorithms, but also on techniques to evaluate the output itself in an \<automatic> fashion. In this field, Togelius et al.\cite{10.1007/978-3-642-12239-2_15} defined \<Search-Based Procedural Content Generation>, a particular kind of \<Generate-And-Test>\footnote{Algorithms with both a generation and an evaluation component, that depending on some criterion, decide to keep the current result or to generate a new one.} algorithm, where the generated content, instead of being just accepted or discarded, is evaluated assigning a suitability \<score> obtained from a \<fitness function>, used to select the best candidates for the next iterations.

% PROCEDURAL CONTENT GENERATION IN FPS %

\section{Procedural Content Generation for FPS maps}

We have really few examples of commercial FPS that use PCG to generate their maps: with the exception of \<Soldier of Fortune II: Double Helix>\footnote{ Raven Software, 2002.}, that employs these techniques to generate whole missions, the few other cases we have are all roguelikes with a FPS gameplay, like \<STRAFE>\footnote{Pixel Titans, 2017.}.

\par

Despite the total lack of FPS games using procedural generation to obtain multiplayer maps, researchers have proved that search-based procedural content generation can be an useful tool in this field. This method has been applied for the first time by Cardamone et al.\cite{Cardamone:2011:EIM:2008402.2008411}, who tried to understand which kind of \<deathmatch>\footnote{A widely used multiplayer game mode where the goal of each player is to kill as many other players as possible until a certain end condition is reached, commonly being a kill limit or a time limit.} maps created the most enjoyable gameplay possible. To achieve this, the authors generates maps for \<Cube 2: Sauerbraten>\footnote{Wouter van Oortmerssen, 2004} by maximizing a fitness function computed on the \<fight time> data collected from \<simulations>\footnote{In the field of search-based procedural generation, fitness function based on simulation are computed on the data collected from a match between artificial agents in the map at issue. They differ from \<direct> and \<interactive> functions, that evaluate, respectively, the generated content and the interaction with a real player.}, with the fight time being the time between the start of a fight and the death of one of the two contenders. The choice of this fitness function is based on the consideration that a long duration of the fight is correlated to the presence of interesting features in the map, such as escape or flanking routes, hideouts and well positioned resources. 

\par

Stucchi\cite{EvoluzioneMappeBilanciate}, yet remaining in the same field, attempted a completely different use of procedural generation, by producing balanced maps for player with different weapons or different levels of skill. For doing so, he generates procedural maps via evolutionary algorithms, evaluating them with a fitness function based on simulation that computed the entropy of kills. Starting from a situation where one of the two players was at a disadvantage, Stucchi is able to prove that changes in the map structure allow to achieve a significant balance increase.

\par

Arnaboldi\cite{SviluppoDiUnFramework} combined these two approaches, creating a framework that automatically produces maps using a genetic process like the one of Cardamone. In Arnaboldi work, however, the fitness function is way more complex, since it considers a high number of gameplay metrics, and the AI of the \<bots>\footnote{The artificial players of a video game.} is more like the one of a human player, thanks to a series of adjustments made to the stock \<Cube 2> one. These improvements significantly increase the scientific accuracy and the overall quality of the output, allowing to identify and analyze some recurring pattern and their relationship with the statistics gathered during the simulation. 

\par

Ølsted et al.\cite{DesignerJob} moved the focus of their research from deathmacth to squad game mods with specific objectives, sustaining not only that the maps generated by Cardamone et al. are not suitable for this kind of gameplay, but also that they not satisfy what they define as \<The Good Engagement> (or \<TGE>) rules, a set of rules that a FPS should satisfy to support and encourage interesting player choices, from which an engaging gameplay should emerge naturally. By analyzing the \<Search \& Destroy>\footnote{A multiplayer game mode where players, divided in two teams, have to eliminate the enemy team or detonate a bomb in their base.} mode of games like \<Counter Strike>\footnote{Valve Software, 2000} and \<Call of Duty>\footnote{Infinity Ward, 2003}, they defined a process to generate suitable maps: starting from a grid, some nodes are selected and connected among them, the result is then optimized to satisfy the TGE rules and finally rooms, resources, objects and spawn points are added, as can be seen in figure \ref{fig:olstedGenerativeProcess}. Opting for an \<interactive> approach, the fitness function used for the evolution of these maps is computed on the binary appreciation feedback expressed by real users, since the authors consider bot behavior too different form the one of real users.

\begin{figure}
  \includegraphics[width=\linewidth]{olstedGenerativeProcess}
  \caption{Visual representation of Ølsted et al.\cite{DesignerJob} generative process.}
  \label{fig:olstedGenerativeProcess}
\end{figure}

\par

A completely different approach from the ones listed above is the one of Anand and Wong\cite{10.1007/978-3-662-45212-7_19}, who employed search-based procedural generation to create \<online>, automatically and rapidly multiplayer maps for the \<Capture and Hold>\footnote{A multiplayer game mode where players, divided in two teams, fight for the control of some strategic areas. The score of each team increases over time proportionally to the number of controlled points until one of the two teams reaches a given limit, winning the game.} game mode, without compromising the quality of the generated maps. To achieve this result, they employ a genetic approach, which fitness function is evaluated directly on the topology the map, considering four different factors: the connectivity between regions, the number of points of collision, the balancing in the positioning of control points and spawn points. With no need to simulate matches, this process can be completed in a matter of seconds. This genetic process starts from three generated maps and the evolution is performed by mutation. To obtain the initial maps, Anand and Wong populate a grid with random tiles, they clean it of undesired artifacts and they identify regions within it, that are then populated with strategic points, resources, spawn points and covers. Despite its good results, this approach is not too sound on a scientific stand point, since it directly depends on the validity of the selected topological metrics and as we have seen it is still not to clear which are the good elements of a level.

\par

Finally, back to deathmatch, Cachia et al.\cite{MultiLevelEvolution} extended search-based procedural generation to multi-level maps, generating the ground floor with one of the methods defined by Cardamone and employing a random digger for the first floor. The final result can be seen in figure \ref{fig:multiLevelEvolution1}. Their algorithm also positions spawn points and resources through a topological fitness function, which implies the same problems described for of Anand and Wong's approach.

\begin{figure}
  \includegraphics[width=\linewidth]{multiLevelEvolution}
  \caption{One of the maps evolved by Cachia's et al.\cite{MultiLevelEvolution} algorithm.}
  \label{fig:multiLevelEvolution1}
\end{figure}

% FPS DESIGN %

\section{History of Level Design in FPS}

% GRAPH THEORY %

\section{Graph Theory in video games}

% SUMMARY %

\section{Summary}

% CHAPTER 3 - UNITY FRAMEWORK %

\chapter{Features of the research framework}

% INTRODUCTION %

In this chapter we describe the \<framework> that we have developed to perform user-based online validation for researches in procedural content generation of multiplayer levels for Firsts Person Shooters. In the first section we give an overview of the framework, of its features and of its components, analyzing them one by one in the following sections.

% FRAMEWORK OVERVIEW 

\section{Overview}

We designed our framework with the objective of providing a valid alternative to the games currently employed as a validation tool in this research field. All the available options, like \<Cube 2: Sauerbraten>, are powerful tools to perform validation via artificial agents, but they are not suitable for user-based validation. A data-collection campaign based on these games requires to download the game or to take part in real-life play-test sessions, but these options discourage potential participants because they are significantly time-consuming. For this reason, we decided to develop a framework that is as light as possible, with a WebGL build weighting less than 10MB that can be played using any browser. The framework was developed with Unity.

\par

Since the purpose of this tool is to be used in research, we decided to support many map representation formats used in previous works and we designed our framework to be as modular, expansible and configurable as possible.

\section{The framework structure}

The framework collects data by assigning to the users \<matches> to play. A match is defined by the \<game mode> and by the \<map type>, which in turn is defined by the \<map topology> and by the \<map appearance>. The \<map topology> defines how the map is going to \<be> and depends on the algorithm used to generate it, whereas the \<map appearance> defines how the map is going to \<look> and depends on how the map is assembled. This implies that the map type defines a whole array of procedurally generated maps that share the same topology and appearance. Therefore, when referring to a match we are considering a specific game-mode played in a procedurally generated map. If needed, it is possible to use a pre-generated map instead of generating a new one, by providing it as input in one of the supported formats. In this case the \<map topology> defines how to interpret the input, that is then displayed considering the \<map appearance>.

\par

A match is defined by combining different modular \<Manager> objects, each of which controls a different aspect of the match. To assure their interchangeability, most modules are defined by their own abstract class.

\subsection{The Game Manager}

The \<Game Manager> is the module responsible for the overall behavior of a match. Each game mode consists in a different implementation of the \<Game Manager>. It leans on the \<Map Manager> for the generation and the assembly of the map and on the \<Spawn Point Manager> for the spawn of entities. The \<Game Manager> controls the life-cycle of the match, that can be divided in the following phases:

\begin{itemize}
\item \<Setup>, all the modules are initialized.
\item \<Generation>, the \<Map Manager> generates or imports the map and assembles it.
\item \<Ready>, the \<Game Manager> displays a countdown announcing the start of the game.
\item \<Play>, the \<Game Manager> handles the game while the \<Experiment Manager> logs the actions of the player, if needed. This phase continues until an end condition is satisfied.
\item \<Score>, the \<Game Manager> stops the game and displays the final score.
\end{itemize}

\subsection{The Spawn Point Manager}

The \<Spawn Point Manager> contains a list of all the spawn points displaced on the map, that is populated during the \<Generation> phase by the \<Game Manager>. When the \<Game Manager> needs to spawn an entity, the \<Spawn Point Manager> provides a random spawn point from the ones that have not been used in a certain amount of time. If no spawn point meets this condition, the extraction is made from the complete pool.

\subsection{The Map Manager}

The \<Map Manager> controls the generation, the import and the assembly of the map and the displacement of objects inside it. It leans on the \<Map Generator> for the generation, on the \<Map Assembler> for the <assembly>\footnote{With \<assembly> we mean the operation of creating a 3D model of the map starting from its matrix representation.}, on the \<Object Displacer> for the \<displacement>\footnote{With \<displacement> we mean the operation of placing the 3D models of the objects in the assembled map, according to their position defined by the \<Map Generator> trough a \<positioning> algorithm.}, whereas it performs the import itself. If the map is provided as input, the \<Map Generator> is not called.

\par

Maps are represented as matrices of characters, where each character corresponds to a tile. Depending on the character, a tile can represent a wall, a room or an object inside a room. If a tile corresponds to a wall we say it is \<filled>, if it corresponds to a room we say it is \<empty>.
 
\subsection{The Map Generator}

The \<Map Generator> controls the generation of the map. Each implementation of the \<Map Generator> defines a different topology depending on the used generation algorithm and on how its parametric setting are tuned. Some of these setting are shared by all the implementations, whereas some of them are implementation-specific.

\par

The shared settings are used to define the size of the map and its encoding, to define the objects and to impose some constraints on their positioning:

\begin{itemize}
\item \<Width>, the width of the matrix that represents the map.
\item \<Height>, the height of the matrix that represents the map.
\item \<ObjectToObjectDistance>, the minimum number of cells that must separate two objects. 
\item \<ObjectToWallDistance>, the minimum number of cells  that must separate an objects and a wall.
\item \<BorderSize>, the width of the border placed all around the map once it has been generated, expressed in number of cells.
\item \<RoomChar>, the character used to represent a clear cell where the player can walk.
\item \<WallChar>,  the character used to represent a filled cell where the player can not walk.
\item \<MapObjects>, a list of the objects that must be placed in the map.
\end{itemize}

The objects contained in \<MapObjects> can represent spawn points, resources or decoration. They have the following properties:

\begin{itemize}
\item \<ObjectChar>, the character used to represent the object.
\item \<NumObjPerMap>,  the number of objects of that kind that must be placed in the map.
\item \<PlaceAnywhere>, if this value is set to true, the restriction on the distance from the walls is ignored.
\item \<PositioningMode>, the algorithm used to position the object in the map.
\end{itemize}

The framework provides three different algorithms to position the objects inside the map:

\begin{itemize}
\item \<Rain>, positions the objects selecting random cells from the ones that are empty and satisfy the  \<ObjectToWallDistance> constraint.
\item \<Rain Shared>, positions the objects selecting random cells from the ones that are empty and satisfy the  \<ObjectToWallDistance> constraint and the \<ObjectToObjectDistance> constraint on the objects that have been placed using \<Rain Shared>.
\item \<Rain Distanced>, positions the objects selecting random cells from the ones that are empty and satisfy the  \<ObjectToWallDistance> constraint and the \<ObjectToObjectDistance> constraint on the objects with the same \<ObjectChar>.
\end{itemize}

We now analyze the available implementations of the  \<Map Generator>.

\subsubsection{Cellular generation}

The \<Cellular Generator> employs a parametric \<cellular automaton>\footnote{A \<cellular automaton> consists of a grid of cells, each in one of a finite number of states, such as on and off. For each cell, a set of cells called its neighborhood is defined, usually composed by the cells that share at least one vertex with it (referred as \<8-neighbors>). Given the current state of the grid, a new generation is created, according to some fixed rule that determines the new state of each cell depending on the current state of the cell itself and of the cells in its neighborhood.} to generate a natural looking map. The algorithm (see Algorithm \ref{alg:cellular}) depends on the following parameters:

\begin{itemize}
\item \<RandomFillPercent>, the percentage of tiles that are randomly filled during the initialization of the algorithm.
\item \<SmoothingInteration>, the number of generations the cellular automaton is ran for.
\item \<NeighbourTileLimitLow>, the minimum number of neighbors a cell must have to became filled.
\item \<NeighbourTileLimitHigh>, the maximum number of neighbors a cell must have to became empty.
\item \<WallThresholdSize>, the minimum number of cells that an isolated filled region must include to not be deleted.
\item \<RoomThresholdSize>, the minimum number of cells that an isolated void region must include to not be deleted.
\item \<PassageWidth>, the width of a passage connecting two different areas, expressed in number of cells.
\end{itemize}

\par

% How these parameters influence the map

% Six different maps with different values

\begin{algorithm}[H]
\label{alg:cellular}
\SetAlgoLined

\For{every cell in the map}{
	empty the current cell\;
}

\While{the percentage of filled cells is less than RandomFillPercent} {
	select a random cell\;
	fill the selected cell\;
}

\For{\#SmoothingInteration iterations} {
	\For{every cell in the map}{
		count the 8-neighbors of the cell\;
		\If{the cell has more than NeighbourTileLimitLow neighbors}{
  			fill the current cell\;
   		}
		\If{the cell has less than NeighbourTileLimitHigh neighbors}{
  			empty the current cell\;
   		} 
	}
}

\For{every connected region of empty cells}{
	\If{the number of cells in the region is less than RoomThresholdSize}{
		fill all the cells in the region\;
	}
}

\For{every connected region of filled cells}{
	\If{the number of cells in the region is less than WallThresholdSize}{
		empty all the cells  in the region\;
	}
}

connect all the regions composed by filled cells\;
place the objects\;

\caption{Cellular generation algorithm}
\end{algorithm}

\subsubsection{Divisive generation}

\subsubsection{Random digger generation}

\subsubsection{Multi-level generation}

\subsection{The Map Assembler}

The \<Map Assembler> controls the assembly of the map. Each different implementation of the Map Assembler corresponds to a different appearance.

\subsection{The Object Displacer}

The \<Object Displacer> associates a character that represents neither a wall or a clear cell to the corresponding object, displacing it in the correct position. During this process, it populates a dictionary containing all the objects in the map divided by category, that is used by the \<Game Manager> to populate the list of spawn points used by the \<Spawn Point Manager>.

% MAP REPRESENTATION %

\section{Map representation}

% WEAPONS AND OBJECTS %

\section{Weapons and objects}

% GAME MODES %

\section{Game modes}

% LOGGING %

\section{Logging}

% EXPERIMENT MANAGEMENT %

\section{Experiment management}


% CHAPTER 4 - PYTHON FRAMEWORK %

\chapter{The graph-based tool}

% INTRODUCTION %

In this chapter we describe the tool that we have developed to perform analysis and populating of pre-generated maps using \<Graph Theory>. After a quick overview, we introduce the analysis capabilities of this tool and then we present how we have employed them, together with Graph Theory, to strategically place resources (including spawn points) in pre-generated maps.

% DESCRIPTION %

\section{Description of the tool}

This tool generates different kind of graphs, starting from the text and the All-Black representation of a map, that are used to perform various analysis and manipulation operations.

\par

The use of All-Black format is convenient, because it provides by default a logical division of the map in different areas and it allows our tool to be applied by other researchers, since as we have seen the All-Black format is widely used in this field. We have used this tool to position objects in a pre-generated map, but, for instance, it could be used to address the identification and definition of design patterns from an unfamiliar perspective or for \<direct evaluation> in Search Based PCG.

\par 

We developed this tool in \<Python>, since it offers many solid Graph Theory libraries, as \<NetworkX>, that is the one we used.

% ANALYSIS %

\section{Analysis of pre-generated maps}

The analysis is performed by generating different kind of undirected graphs, each one used to highlight a different feature of the map in question.

\subsection{Outlines graph}

The \<outlines graph> is generated starting from the All-Black representation of a map and is obtained by associating a node to every vertex of every room and corridor and by connecting the non-adjacent ones that belong to the same outline. This graph has a single kind of node (\<vertex node>) that contains the coordinates of the tile it represents, which are used to position the node when the graph is visualized. Figure \ref{img:graph_out} shows an example of this graph.

\par

This graph can be used to visualize the rooms which compose the map.

\subsection{Reachability graphs}

Our tool can generate various kinds of \<reachability graphs> that represent various ways in which an entity can navigate a map. In these graphs a node represents a position that an entity can reach, whereas an edge indicates a viable path from a position to another.

\subsubsection{Tiles graph}

The \<tiles graph> is generated starting from the text representation of a map and is obtained by associating a node to each empty tile and by connecting each node to its corresponding 8-neighbors. The horizontal and vertical edges have cost $1$, whereas the diagonal ones have cost $\sqrt{2}$. This graph has a single kind of node (\<tile node>) that contains the coordinates of the tile it represents, which are used to position the node when the graph is visualized. Figure \ref{img:graph_tile} shows an example of this graph.

\par

This graph can be used to find the minimum distance that separates two cells, along with the shortest path that connects them.

\subsubsection{Rooms graph}

The \<rooms graph> is generated starting from the All-Black representation of a map and is obtained by associating a node to each room and corridor and by connecting nodes which corresponding rooms or corridors overlap, using as weight the Euclidean distance of their central tile. This graph has a single kind of node (\<room node>) used to represent both rooms and corridors that contains the coordinates of the closest and furthest vertex of the room from the origin. When visualized, each node is positioned on the coordinates of the central tile of the room it represents. Figure \ref{img:graph_room} shows an example of this graph.

\par

This graph can be used to analyze the topology of a map, in order to find loops, choke points, central areas and other kind of structures.

\subsubsection{Rooms and resources graph}

The \<rooms and resources graph> is an extension of the room graph, which also include resources as nodes, that are connected to the nodes corresponding to the rooms and corridors which contain them. In addition to the room node inherited form the rooms graph, this graph has a node to represent resources (\<resource node>) that contains the coordinates of the resource, which are used to visualize the node, and the character associated to the resource. Figure \ref{img:graph_room_res} shows an example of this graph.

\subsection{Visibility graph}

The \<visibility graph> is generated starting from the text representation of a map and is obtained by associating a node to each empty tile and by connecting each node to all the tiles that are visible from that node. For two tiles to be respectively visible, it must be possible to connect them with a line without crossing any filled tile. This graph has a single kind of node (\<visibility node>) that contains the coordinates of the tile it represents, which are used to position the node when the graph is visualized, and its \<degree centrality>, i.e. the number of edges incident to that node. 

\par

To make this graph easier to read by the user, the tool associates a color to the nodes, which ranges from blue, for the ones with the minimum visibility, to red, for the ones with the maximum visibility. This can be seen in figure \ref{img:graph_visibility}.

\par

This graph can be used to analyze which areas of the map are more exposed and which ones are more repaired.

\begin{figure}[]
	\centering
  	\begin{subfigure}[t]{0.45\linewidth}
		\includegraphics[width=\linewidth]{graph_divisive}
     		\caption{The map.}
     		\label{img:graph_divisive}
 	\end{subfigure}
  	\begin{subfigure}[t]{0.45\linewidth}
    		\includegraphics[width=\linewidth]{graph_out}
    		\caption{The outlines graph of the map.}
     		\label{img:graph_out}
  	\end{subfigure}
  	\begin{subfigure}[t]{0.45\linewidth}
    		\includegraphics[width=\linewidth]{graph_tile}
    		\caption{The tiles graph of the map.}
     		\label{img:graph_tile}
  	\end{subfigure}
  	\begin{subfigure}[t]{0.45\linewidth}
    		\includegraphics[width=\linewidth]{graph_room}
    		\caption{The rooms graph of the map.}
     		\label{img:graph_room}
 	\end{subfigure}
  	\begin{subfigure}[t]{0.45\linewidth}
    		\includegraphics[width=\linewidth]{graph_room_res}
    		\caption{The rooms and resources graph of the map.}
     		\label{img:graph_room_res}
  	\end{subfigure}
  	\begin{subfigure}[t]{0.45\linewidth}
    		\includegraphics[width=\linewidth]{graph_visibility}
    		\caption{The visibility graph of the map.}
     		\label{img:graph_visibility}
  	\end{subfigure}	
	\caption{A map and all the graphs that the tool can generate from it.}
\end{figure}

\subsection{Interesting metrics}

Considering the graphs, in particular the ones with room nodes, the following metrics defined by Graph Theory provide interesting information about the layout of a map:

\begin{itemize}
\item \<Degree centrality>: defined for a node, it is the number of edges that the node has. If the node represents a room, it measures how many entrance or exits the room has. 
\item \<Closeness centrality>: defined for a node, it measures its centrality in the graph, computed as the sum of the lengths of the shortest paths between the node and all other nodes in the graph. If the node represents a room, it measures how central the room is.
\item \<Betweenness centrality>: defined for a node, it measures its centrality in the graph, computed as the number of shortest paths connecting the nodes that pass through the node. If the node represents a room, it measures how central the room is.
\item \<Connectivity>: defined for a graph, it is the minimum number of elements (nodes or edges) that need to be removed to disconnect the remaining nodes from each other. If the graph represents a map, it measures the existence of isolated areas.
\item \<Eccentricity>: defined for a node, it is the maximum distance from the node to all other nodes in the graph. If the node represents a room, it measured how isolated the room is.
\item \<Diameter>: defined for a graph, it is the maximum eccentricity of its nodes. If the graph represents a map, it measures the size of the map.
\item \<Radius>: defined for a graph, it is the minimum eccentricity of its nodes. If the graph represents a map, it measures how distanced the rooms are from each other.
\item \<Periphery>: defined for a graph, it is the set of nodes with eccentricity equal to the diameter. If the graph represents a map, it defines its peripheral areas.
\item \<Center>: defined for a graph, it is the set of nodes with eccentricity equal to the radius. If the graph represents a map, it defines its central areas.
\item \<Density>: defined for a graph, it ranges from 0 to 1, going from a graph without edges to a complete graph. If the graph represents a map, it measures how complex it is.
\end{itemize}

% POPULATING %

\section{Populating of pre-generated maps}

We have defined multiple heuristics to populate a map with spawn points and other resources using the metrics that can be extracted from a graph. These heuristic are a mathematical transposition of rules and patterns concerning resource placement that we have extracted from the work of Tim Schäfer\cite{great1vs1}, who has performed an in depth analysis of multiplayer 1vs1 maps for \<Quake 2>\footnote{Id Software, 1997}.

\subsection{Rules and patterns for resources positioning}

The balance of a deathmatch game radically changes each time that a player is killed. If the game has more than two players, the player who won the fight does not gain any strategic advantage, since he still has the other players to face, whereas the defeated player is put at considerable disadvantage, because on death he loses all the weapons and ammunition that he has collected. In a 1vs1 match, a kill has an even stronger influence, since the surviving player has more weapons and ammunition and gains the complete control of the map, that comes with the chance of scoring another easy kill, as soon as the other player respawns, or of searching for additional equipment. Schäfer refers to the surviving player as \[up-player] and to the defeated one as \[down-player].

\par

To obtain a multiplayer map that is interesting and fun to play, it is important to consider the up-player vs down-player dynamic both when defining the map layout and when positioning resources.

\par

The spawn points, i.e. the locations where the down-player reappears, should be positioned in areas that are of low interest for the up-player and that are easy to leave. Obviously, central hubs and dead ends are a bad choice, whereas rooms with 2 or 3 exits are usually the best option. 

\par

For what concerns the resources, they must be placed considering both the up-player vs down-player dynamic and the characteristics of the resource itself. It is important to place the right amount of resources on the map, because too many would eliminate the need for exploration, whereas too few would disadvantage the down player. It is also important not to place too many powerful items in the same area or in boring spots, since the risk to obtain them should always be proportional to the strategical advantage they allow to achieve. It is important to consider that a powerful resource is interesting for both players, so it often acts as a \<point of collision>. The resources usually are of five kinds: health packs, \<armors>, \<power-ups>, ammunition and weapons. 

\par

The health packs are placed in zones that are safe or not too dangerous. They have no use for the down-player, that respawns with full health, but they can be useful for the up-player, if he has been damaged during the fight, whereas they always come in handy during a fight or after that one of the contenders disengages.

\par

Armors, which supply a second health that is consumed before the main one, are usually placed in spots that are aimed both at the down-player and at the up-player: objects that provide a small quantity of armor should be easy to achieve, whereas the ones that provide full armor should be placed in dangerous areas.

\par

Power-ups grant temporary advantages to the player who collects them, like invisibility or increased damage, and are placed in locations difficult to reach and contextual to their effect.

\par 

The position of a weapon and of its ammunition depends on the weapon itself. We can divide the weapons in three categories: \<weak>, \<medium> and \<strong>. Weak weapons are of a certain interest for the down-player, if he has not collected any other weapon yet, and of no interest for the up-player, so they are placed near spawn-points or in gaps where no other weapon is available, together with their ammunition. Medium weapons are of high interest for the down-player, since he needs to get one of them as soon as possible if he wants to face the up-player, so they are placed in areas that are easy to reach and the same goes for their ammunition. Finally, the strong weapons should be placed in areas that are strategically disadvantageous, like dead ends or vertically dominated areas, or difficult to reach. If a weapon is very contextual, i.e. it is useful in very few situations, it is usually placed in an area that allows to take advantage of its features, whereas a weapon that is strong in almost any situation is usually placed in an area where it cannot be used optimally (e.g. a rocket launcher in a small room).

\subsection{Placement process}
\label{ss:placement}

Starting from these considerations, we defined a process based on heuristics that allows to position any kind of resource. This process consists in two phases: the selection of a room and the selection of a tile inside the room.

\par

The selection of a room is performed considering two suitability criteria:

\begin{itemize}
\item \<Degree suitability>: defined by the function $D(r)$, where $r$ is a room node, it measures how much the degree centrality of the node matches the desired one.
\item \<Resources closeness suitability>: defined by the function $C_{res}(r)$, where $r$ is a room node, it measures how much the closeness of the node to the already placed resource nodes matches the desired one.
\end{itemize}

Given $G$ the rooms and corridors graph and $R \subset G$ the subset of room nodes, the room node which is selected is the one which maximizes the following weighted sum of functions:

\begin{align}
room = \argmax_{r \in R} (w_D  \times D(r) + w_{C_{res}}  \times C_{res}(r))
\end{align}

\noindent
Both functions should be defined to output a value in the range $[0,1]$.

\par

Once that the room is selected, the selection of a tile is performed considering three suitability criteria:

\begin{itemize}
\item \<Visibility suitability>: defined by the function $v(t)$, where $t$ is a tile node, it measures how much the visibility of the node matches the desired one.
\item \<Wall closeness suitability>: defined by the function $c_{wall}(t)$, where $t$ is a tile node, it measures how much the proximity of the corresponding tile to the walls matches the desired one.
\item \<Resources closeness suitability>: defined by the function $c_{res}(t)$, where $t$ is a tile node, it measures how much the proximity of the node with already placed resource nodes matches the desired one.
\end{itemize}

Given $G$ the tiles graph and $T \subset G$ the subset of tiles contained by the selected room, the tile which is selected is the one which maximizes the following weighted sum of functions:

\begin{align}
room = \argmax_{t \in T} (w_v \times v(t) + w_{c_{wall}}  \times c_{wall}(t) + w_{c_{res}}  \times c_{res}(t))
\end{align}

\noindent
All three functions should be defined to output a value in the range $[0,1]$.

\par

The process can be repeated as many time as needed, after having updated the graphs with the newly added resource. To obtain the best result, the resources should be placed in the order their heuristics are presented.

\subsection{Spawn points placement}

We have defined three heuristics for the placement of spawn points, the first one follows the process defined in subsection \ref{ss:placement}, the other two do not.

\subsubsection{Low risk heuristic}

This method selects rooms that have a small number of connections, are not dead ends and are distant from each other and for each one of them places the spawn point on the tile that offers the best balance between low visibility and distance from the walls. In this way spawn points are placed in passageways that are sheltered and easy to leave. 

\par

Given $G$ the rooms and corridors graph, $R \subset G$ the subset of room nodes, $S \subset G$ the subset of resource nodes (composed exclusively by the already positioned spawn points), the most suitable room for containing a spawn point is selected using

\begin{align}
\label{eq:lowriskdeg}
D(r) = \begin{cases}
    		\hfil 0 & \text{if } \degcent((r)) = 1 \\
    		1 - \cfrac{\degcent(r) - \min_{r' \in R}\degcent(r')}{\max_{r' \in R}\degcent(r') - \min_{r' \in R}\degcent(r')} & \text{if } \degcent(r) \neq 1 \
  	\end{cases}
\end{align}

\begin{align}
\label{eq:lowriskres}
C_{res}(r) = \min_{n \in G}
  	\begin{cases}
    		\hfil 1 & \text{if } n \notin S \\
    		\cfrac{\spl(r, n)}{\diam(G)} & \text{if } n \in S \
  	\end{cases}
\end{align}

\noindent
where $deg$ denotes the connectivity degree of a node. Equation \ref{eq:lowriskdeg} promotes rooms with few passages but that are not dead ends, whereas equation \ref{eq:lowriskres} promotes rooms that are distant from the already placed spawn points. Both are normalized in the range $[0,1]$. We experimentally found that the weights which provide the best results are $w_D = 1 $ and $ w_{C_{res}} = 0.5 $.

\par

Given G the visibility graph, $T \subset G$ the subset of tile nodes that belong to the room, $R \subset G$ the subset of tile nodes which contain a resource (composed exclusively by the already positioned spawn points), the most suitable tile for containing a spawn point is selected using

\begin{align}
\label{eq:lowvis}
v(t) = 1 - \cfrac{\degcent(t) - \min_{t' \in G}\degcent(t')}{\max_{t' \in G}\degcent(t') - \min_{t' \in G}\degcent(t')}
\end{align}

\begin{align}
\label{eq:lowwallclos}
c_{wall}(r) = \cfrac{\walld(t, room)}{ \maxwalld(room)}
\end{align}

\begin{align}
\label{eq:lowresclos}
 c_{res}(r) = \begin{cases}
    		\hfil 0 & \text{if } | S | = 0 \\
    		\min_{s \in S}\cfrac{\cartd(t, s)}{d}& \text{if }  | S | > 0 \
  	\end{cases} 
\end{align}

\noindent
where \<wall distance> is the distance of a tile from the walls of the room, computed as the sum of the minimum distances from the horizontal and vertical walls, and \<maximum wall distance> is the wall distance of the central tile of the room. Equation  \ref{eq:lowvis} promotes tiles with low visibility, equation \ref{eq:lowwallclos} promotes tiles that are distant from the walls, whereas equation \ref{eq:lowresclos} promotes tiles that are distant from the already placed spawn points. All three are normalized in the range $[0,1]$. We experimentally found that the weights which provide the best results are $ w_v = 1 $ , $ w_{c_{wall}} = 0.5 $ and $ w_{c_{res}}  = 0.5 $.

\subsubsection{Uniform heuristic}

This method selects rooms that are uniformly distributed in the map and for each one of them places the spawn point on the tile that offers the best balance between low visibility and distance from the walls. 

\par

The first room is selected at random, then, given $G$ the rooms and corridors graph, $R \subset G$ the subset of room nodes, $S \subset G$ the subset of resource nodes (composed exclusively by the already positioned spawn points), the remaining rooms are selected with the following heuristic:

\begin{align}
	room = \argmax_{r \in R} ( \min_{s \in S} (\spl(r, s) ))
\end{align}

Once that the room has been found, one of its tiles is selected using equations \ref{eq:lowvis}, \ref{eq:lowwallclos} and \ref{eq:lowresclos} with weights  $ w_{c_{wall}} = 0.5 $ and $ w_{c_{res}}  = 0.5 $.

\subsubsection{Random heuristic}

This method selects rooms at random and for each one of them places the spawn point on the tile that offers the best balance between low visibility and distance from the walls, using using equations \ref{eq:lowvis}, \ref{eq:lowwallclos} and\ref{eq:lowresclos} with weights  $ w_{c_{wall}} = 0.5 $ and $ w_{c_{res}}  = 0.5 $.

\subsection{Health packs placement}

Heath packs are placed in rooms that have a low-medium number of connections and are distant from each other on the tile that offers the best balance between medium visibility and distance from the walls. In this way the heath packs are placed in areas that are easy to reach and are not too exposed.

\par

Given $G$ the rooms and corridors graph, $R \subset G$ the subset of room nodes, $S \subset G$ the subset of resource nodes (composed by the already positioned spawn points and health packs), the most suitable room for containing a health pack is selected using

\begin{align}
\label{eq:lowriskdegh}
D(r) = \cfrac{f_{deg}(r) - \min_{r' \in R}f_{deg}(r')}{\max_{r' \in R}f_{deg}(r') - \min_{r' \in R}f_{deg}(r')} 
\end{align}

\begin{align}
\label{eq:lowriskresh}
C_{res}(r) = \min_{n \in G}
  	\begin{cases}
    		\hfil 1 & \text{if } n \notin S \\
    		\cfrac{\spl(r, n)}{\diam(G)} & \text{if } n \in S \
  	\end{cases}
\end{align}

\noindent
with

\begin{align}
\label{eq:degfit}
f_{deg} = \intd(\degcent(r),0.3,0.5)
\end{align}

\begin{align}
\label{eq:intervaldistance}
\intd(v, v_{min}, v_{max}) = | ( |v_{min}| - |v | ) | + | (|v_{max}| - |v | )| 
\end{align}

\noindent
where \<interval distance> is the distance of a value form the extremes of an interval and it is used to estimate how a value fits an interval. Equation \ref{eq:lowriskdegh} promotes rooms with a low-medium number of passages, whereas equation \ref{eq:lowriskresh} promotes rooms that are distant from the already placed resources. Both are normalized in the range $[0,1]$. We experimentally found that the weights which provide the best results are $w_D = 1 $ and $ w_{C_{res}} = 0.5 $.

\par

Given G the visibility graph, $T \subset G$ the subset of tile nodes that belong to the room, $R \subset G$ the subset of tile nodes which contain a resource, the most suitable tile for containing a health pack is selected using

\begin{align}
\label{eq:lowvish}
v(t) = 1 - \biggg| 0.5 - \cfrac{\degcent(t) - \min_{t' \in G}\degcent(t')}{\max_{t' \in G}\degcent(t') - \min_{t' \in G}\degcent(t')} \biggg| 
\end{align}

\noindent
that promotes tiles with medium visibility. Equations \ref{eq:lowwallclos} and \ref{eq:lowresclos} are used for $c_{wall}(r)$ and $c_{res}(r)$ , respectively. All three are normalized in the range $[0,1]$. We experimentally found that the weights which provide the best results are $ w_v = 1 $, $ w_{c_{wall}} = 0.25 $ and $ w_{c_{res}}  = 0.5 $.

\subsection{Ammunition placement}

Ammunition is placed in rooms that have either a low-medium or a high number of connections and are distant from each other on the tile that offers the best balance between high visibility and distance from the walls. In this way ammunition is placed in areas that are either easy or difficult to reach and easy to spot.

\par

Given $G$ the rooms and corridors graph, $R \subset G$ the subset of room nodes, $S \subset G$ the subset of resource nodes (composed by the already positioned spawn points, health packs and ammunition), the most suitable room for containing ammunition is selected using equations \ref{eq:lowriskresh} and \ref{eq:lowriskdegh}, with

\begin{align}
\label{eq:degfita1}
f_{deg} = \intd(\degcent(r),0.2,0.4)
\end{align}

\noindent
for obtaining rooms with a low-medium number of connections and
 
\begin{align}
\label{eq:degfita2}
f_{deg} = \intd(\degcent(r),0.8,0.9)
\end{align}

\noindent
for obtaining rooms with a high number of connections. We experimentally found that the weights which provide the best results are $w_D = 1 $ and $ w_{C_{res}} = 0.25 $.

\par

Given G the visibility graph, $T \subset G$ the subset of tile nodes that belong to the room, $R \subset G$ the subset of tile nodes which contain a resource, the most suitable tile for containing ammunition is selected using

\begin{align}
\label{eq:highvisa}
v(t) = \cfrac{\degcent(r) - \min_{r' \in R}\degcent(r')}{\max_{r' \in R}\degcent(r') - \min_{r' \in R}\degcent(r')}
\end{align}

\noindent
that promotes tiles with high visibility. Equations \ref{eq:lowwallclos} and \ref{eq:lowresclos} are used for $c_{wall}(r)$ and $c_{res}(r)$, respectively. All three are normalized in the range $[0,1]$. We experimentally found that the weights which provide the best results are $ w_v = 1 $, $ w_{c_{wall}} = 0.25 $ and $ w_{c_{res}}  = 0.5 $.

\subsection{Weapon placement}

For what concerns the weapons provided by the framework we have not defined any specific function yet, but, beside the assault rifle which is always available to the player, they could be placed as follows:

\begin{itemize}
\item \<Shotgun>: since it is a medium damage weapon it could be positioned in a room that has a medium number of connections and is relatively close to a spawn point. 
\item \<Rocket launcher>: since it is a high damage weapon it could be positioned in a room that has a high number of connections, creating an interesting collision point, or in a dead end, where its utility is limited.
\item \<Sniper rifle>: since it is a high damage weapon it could be positioned in a room that has a high number of connections, creating an interesting collision point.
\end{itemize}

% SUMMARY %

\section{Summary}

In this chapter we analyzed the tool that we have developed to perform map analysis and populating using Graph Theory. After an overview of the graphs and metrics that allow to highlight interesting information about a map, we listed the rules and patterns commonly used to position resources in deathmatch maps and we described how we converted them into heuristics.

% CHAPTER 5 - FIRST EXPERIMENT %

\chapter{Experiment on spawn points placement heuristics}

% INTRODUCTION %

In this chapter we describe the experiment that we have performed for validating the placement heuristics for spawn points described in the previous chapter and, at the same time, for testing the data-collection capabilities of our framework, that was used to setup and manage the experiment.

% DESCRIPTION %

\section{Description}

With this experiment we analyzed how the placement of spawn points influences the up-player vs down-player dynamic. The experiment tries to recreate the situation where the up-player, once killed his opponent, tries to find him as soon as possible, just after his respawn, to score another easy kill. As we have seen, a well-designed map should slow down this operation by having its spawn points in areas that are not central, are easy to leave and covered. The \<low risk> heuristic described in chapter \ref{sss:lowrisk} has been designed to place the spawn points according to such criteria, so, to prove its effectiveness, we decided to confront how this facet of the up-player vs down-player dynamic changes with respect to spawn points placed with the \<uniform> heuristic, defined as well in chapter \ref{sss:uniform}, which selects the areas to contain spawn to be distributed evenly inside the map.

\par

To highlight this particular dynamic, we designed a game mode where the user, which represents the up-player, must find and destroy a static target, which represents the down-player, as many times as possible before times runs out. Each time that the user destroys a target, it respawns at a random spawn point. The user cannot die and has infinite ammunition, so he does not have to look for resources.

% SETUP %

\section{Setup}

For this experiment, we setup the \<Experiment Manager> to propose in each play session a quick tutorial, two matches and a survey. The experiment was composed by three \<studies>, corresponding to different pre-generated maps, each one composed by two \<cases>, one that corresponded to the \<low risk> spawn points distribution and one that corresponded to a pool of five \<uniform> spawn point distributions. In a play session, the user played the same map twice, once with the low risk distribution and once with one of the uniform distributions, in a random order and with the map flipped in one of the two matches. We used the survey to profile how much the user was familiar with video games and FPS, to evaluate his skill and to get a feedback about in which of the two maps it was harder to find the targets. The experiment was deployed online and played by the users via browser on their own computer.

\par

Each match had \<Target Hunt> as game mode. The duration of the match was set to three minutes, the list of spawnable entities consisted of just one target and the only weapon available to the player was the \<assault rifle> with infinite ammunition. The pre-generated maps were stored as text files and were loaded by the \<Divisive Generator> and displayed with the \<Prefab Assembler>. 

\par
 
For each match, a complete game log was saved, along with the following performance metrics, saved in a separate log:
 
 \begin{itemize}
\item \<TargetLogs>: this field contains a list of all the targets that the user managed to destroyed. Each entry contains a timestamp of when the target was destroyed, the coordinates of the target, the coordinates of the user, the distance covered by the user and the time passed during the lifespan of the target.
\item \<Shots>: the total number of projectiles shot by the user.
\item \<Hits>: the number of projectiles that hit a target.
\item \<Accuracy>: the percentage of projectiles that hit a target.
\item \<Kills>: the total number of targets destroyed by the user.
\item \<Distance>: the total distance covered by the user during the match, considering cells of unitary width.
\item \<AvgKillTime>: the duration of the match divided by the number of kills.
\item \<AvgKillDistance>: the total distance divided by the number of kills.
\end{itemize}

\noindent
The answers to the survey were saved as well.
 
 \par
 
The performance of the player is measured by \<AvgKillTime>, that is also an indicator of how difficult it is to find targets in the map.

\par
  
We have selected three procedurally generated maps that present radically different layouts and we populated each one of them using both the low risk and the uniform heuristic:

\begin{itemize}
\item \<Arena>: this map presents a wide open arena, two sides of which are adjacent to parallel corridors with many openings. As the visibility heatmap in figure \ref{img:arena_visibility} shows, the central arena allows to control most of the map, whereas the corridors offer some repair and perfect spots to place spawn points. Figure \ref{img:arena_safe} shows the spawn points positioned using the low risk heuristic, whereas figure \ref{img:arena_uniform} shows one of the five configuration produced using the uniform heuristic.
\item \<Corridors>: this map presents many small rooms connected by long corridors. As it can be seen in figure \ref{img:corridors_visibility}, there is no area that allows to control the others and the only point with high visibility are the ones where corridors intersect. Figure \ref{img:corridors_safe} shows the spawn points positioned using the low risk heuristic, whereas figure \ref{img:corridors_uniform} shows one of the five configuration produced using the uniform heuristic.
\item \<Intense>: compared to the previous two, this map presents an intermediate layout, since it has both open areas and small rooms connected by corridors. As it can be seen in figure \ref{img:intense_visibility}, this reflects also on the visibility, that is high in the central open arenas and low in the remaining sections of the map. Figure \ref{img:intense_safe} shows the spawn points positioned using the low risk heuristic, whereas figure \ref{img:intense_uniform} shows one of the five configuration produced using the uniform heuristic.
\end{itemize}

\par

It is important to observe that the low risk and the uniform heuristics select rooms to place a spawn point with two different criteria, but they employ the same logic when selecting a tile inside a room. This means that the two heuristics select tiles which have similar visibility conditions, so the player's performance difference depends exclusively on how the rooms have been selected. This is observable in figures \ref{img:arena}, \ref{img:corridors} and \ref{img:intense}: when the uniform heuristics happens to select a room that has been selected also by the low risk one, the spawn point is placed exactly on the same tile.

\begin{figure}[tp]
	\centering
	\begin{subfigure}[t]{0.3\linewidth}
    		\includegraphics[width=\linewidth]{arena_visibility}
     		\caption{Heatmap showing the visibility of the level.}
		\label{img:arena_visibility}
  	\end{subfigure}  	
  	\hfil
  	\begin{subfigure}[t]{0.3\linewidth}
    		\includegraphics[width=\linewidth]{arena_safe}
     		\caption{Spawn points (in red) placed using the safe heuristic.}
     		\label{img:arena_safe}
  	\end{subfigure}
  	\hfil
  	\begin{subfigure}[t]{0.3\linewidth}
    		\includegraphics[width=\linewidth]{arena_uniform}
     		\caption{Spawn points (in red) placed using the uniform heuristic.}
		\label{img:arena_uniform}
  	\end{subfigure}  
	\caption{``Arena'' map used in the experiment.}
	\label{img:arena}
\end{figure}

\begin{figure}[tp]
	\centering
  	\begin{subfigure}[t]{0.3\linewidth}
    		\includegraphics[width=\linewidth]{corridors_visibility}
     		\caption{Heatmap showing the visibility of the level.}
		\label{img:corridors_visibility}
  	\end{subfigure}  	  
  	\hfil
  	\begin{subfigure}[t]{0.3\linewidth}
    		\includegraphics[width=\linewidth]{corridors_safe}
     		\caption{Spawn points (in red) placed using the safe heuristic.}
     		\label{img:corridors_safe}
  	\end{subfigure}
  	\hfil
  	\begin{subfigure}[t]{0.3\linewidth}
    		\includegraphics[width=\linewidth]{corridors_uniform}
     		\caption{Spawn points (in red) placed using the uniform heuristic.}
		\label{img:corridors_uniform}
  	\end{subfigure}	
	\caption{``Corridors'' map used in the experiment.}
	\label{img:corridors}
\end{figure}

\begin{figure}[tp]
	\centering  	
  	\begin{subfigure}[t]{0.3\linewidth}
    		\includegraphics[width=\linewidth]{intense_visibility}
     		\caption{Heatmap showing the visibility of the level.}
		\label{img:intense_visibility}
  	\end{subfigure}  	
  	\hfil
  	\begin{subfigure}[t]{0.3\linewidth}
    		\includegraphics[width=\linewidth]{intense_safe}
     		\caption{Spawn points (in red) placed using the safe heuristic.}
     		\label{img:intense_safe}
  	\end{subfigure}
  	\hfil
  	\begin{subfigure}[t]{0.3\linewidth}
    		\includegraphics[width=\linewidth]{intense_uniform}
     		\caption{Spawn points (in red) placed using the uniform heuristic.}
		\label{img:intense_uniform}
  	\end{subfigure}
	\caption{``Intense'' map used in the experiment.}
	\label{img:intense}	
\end{figure}

% RESULTS %

\section{Results}

\begin{figure}
 	\centering
  	\begin{subfigure}[t]{0.8\linewidth}
    		\includegraphics[width=\linewidth]{bar_lowrisk}
     		\caption{Low risk heuristic.}
		\label{img:bar_lowrisk}
  	\end{subfigure}  	
  	\begin{subfigure}[t]{0.8\linewidth}
    		\includegraphics[width=\linewidth]{bar_uniform}
     		\caption{Uniform heuristic.}
     		\label{img:bar_uniform}
  	\end{subfigure}
	\caption[Distribution of the matches by number of kills for the two heuristics.]{Distribution of the matches by number of kills for the two heuristics. The matches are also classified by map.}
	\label{img:match_distribution}	
\end{figure}

The data collection campaign for this experiment produced a dataset composed by 27 samples, which distribution between the three maps can be seen in figure \ref{img:match_distribution}. Of 27 samples, 10 are pairs of matches played in map ``Arena'', 9 are pairs of matches played in map ``Corridors'' and 8 are pairs of matches played in map ``Intense''. 

\par

As we have seen, the performance of the player is measured by \<AvgKillTime> and by computing its mean value we observed that the users performed better in the matches associated to the uniform distribution with respect to the ones associated to the the low risk distribution. For the former the mean value of \<AvgKillTime> is $16.06$ seconds and the one of \<kills> is $11.88$ kills, whereas for the latter  the mean value of \<AvgKillTime> is $20.24$ seconds and the one of \<kills> is $10.06$ kills. The increase of 4 seconds on the average time needed to find a target confirms that spawn points placed with the low risk heuristic are more difficult to find and thus such heuristic is valid.

\par

To prove the statistic soundness of this result, we performed the \<Wilcoxon> statistical test\footnote{The \<Wilcoxon signed-rank test> is a \<non-parametric statistical hypothesis test> used to compare two matched samples to assess whether their population mean ranks differ.} by \<Pratt>\footnote{With respect to the standard Wilcoxon test, the one by Pratt considers also the observations for which the difference of the elements in the pair is zero. Since we have 4 cases where the number of kills is the same, we opted for this approach.}, using as matched samples the number of kills in maps with low risk heuristic and the number of kills in maps with uniform heuristic. The test was passed with $\alpha < 0.005$, one-tiled.

\begin{figure}[h]
 	\centering
  	\begin{subfigure}[t]{0.48\linewidth}
    		\includegraphics[width=\linewidth]{scatter_kills}
     		\caption{Kills.}
		\label{img:scatter_kills}
  	\end{subfigure}
  	\hfill
  	\begin{subfigure}[t]{0.48\linewidth}
    		\includegraphics[width=\linewidth]{scatter_avg}
     		\caption{AvgKillDistance.}
     		\label{img:scatter_avg}
  	\end{subfigure}
	\caption[Experiment outcomes by metrics.]{Experiment outcomes by metrics. The outcome associated to the low risk heuristics is on the horizontal axis, the one associated to the uniform heuristics is on the vertical axis.}
	\label{img:metrics}	
\end{figure}

The effect of the heuristics on the metrics that we have defined can be observed by plotting their values, assigning to the horizontal axis the value of the metric in maps populated with the low risk heuristic and to the vertical axis the value of the metric in maps populated with the uniform heuristic. Each point of such graph represents the outcome of a test whose coordinates are the value of the metric in the two matches. By tracing the bisector, it is easy to see for which of the two heuristics a metric is higher. If the points are scattered under the bisector, it means that the metric tends to be higher for the low risk heuristic, if they are scattered around the bisector it means that there is no significant difference, if they are scattered above the bisector it meas that the metric tends to be higher for the uniform heuristic. Figure \ref{img:scatter_kills} shows how the heuristics influence the number of kills: of 27 tests, 17 have more kills in the map  populated using the uniform heuristic, 6 have more kills in the map populated using the safe heuristic and 4 have the same number of kills. Figure \ref{img:scatter_distance} shows how the heuristics influence the average distance covered to find a target, which tends to be higher in maps whose spawn points have been placed using the low risk heuristics.

\par

The result of the uniform are more sèreaded also beacuse the maps are taken dorm a pool. If we analyze the single maps, we can see that with the low risk distribution the arena has mean 10.75, the corridors 10.44 and intense 8.75. The result ofa rena was expected, thanks to its central area that alows to easly control the rest of the map, and also the targets in corridosr to be harder to find than in arena, but would have expected to have intense in between the two, instead it relvealed to be the map where targets were harder to find, probably beacuse its tangled structure. If we consider the uniform heuristic we see an evident increase in arena, 11.88 that was contrary to our expecation, since we expected the situation to remain more or less the same, corridors, instead keeps the kill count almost the same 10.67, whereas intense has a dramatic increase to 12.75. A possible explenation is that corridors ha a really uniform structure, and once in a corridor you can controll all of that portion of the map, so there is no difference in were palcing the spawn point. Instead in arena, with the central area that dominates everything, there is a great difference if it is covered. The same is probably true with Intense, that sums to this the fact htat there are peripheal areal that slow down the serach if a spawn point is tactically palcd inside it.

\begin{figure}
 	\centering
	\includegraphics[width=0.8\linewidth]{percived}
	\caption{Comparison between the effective and the perceived difficulty.}
	\label{img:percived}		
\end{figure}

A final interesting observation concerns how the difficulty was perceived by the user, this cna be seen in figure. At the end of the experiment the user were asked to tell whihc of the two map they played was more difficult: if the first, the second or the were the same. We can see that generally the user percived as more difficult the map were they made more kill, or not perviced any difference at all. Sometimes, altohigh, palters rated as more difficult the map were they èerformed better.

\par

Finally, the comment of the player revealed that the intuition of flipping the map was a good one, since often they didn't recongnised that the two maps were the same.

% CONCLUSIONS %

\section{Conclusions}



% CHAPTER 6 - SECOND EXPERIMENT %

\chapter{Conclusions}

% CONCLUSIONS %

The purpose of this thesis was to create a framework to perform research in procedural content generation for First Person Shooters. Past works have employed open source games, like Cube 2, that allow to perform validation via artificial agents but present many limitations when it is needed to collect information from real users. There was therefore the need of a way to collect data online, in an easy and quick way, and we answered to it by designing our framework to deploy browser playable experiments, that once defined collect data automatically.

\par

Being aimed at research, we wanted our framework to be as versatile as possible, so we opted for a modular and parametric design that is easy to customize and we included many generation algorithms and map representation formats that have been used in previous works. We included the All Black format, defined by Cardamone et al.\cite{Cardamone:2011:EIM:2008402.2008411}, that is a standard in the literature, but we extended it to be more complete and flexible, introducing variable genome size, game elements codification and multi-level support. Because we expect multi-level generation to be a topic of great interest in the years to come, we included the multi-level map format defined by Cachia et al.\cite{MultiLevelEvolution}, who were the first to evolve multi-level maps, but we also extended the classic All-Black representation to be multi-level. This, together with the possibility of including the position of game elements and stairs, should allow to perform evolution of multi-level maps starting from what has already been done in single level research. 

\par

Moreover, we explored how Graph Theory can be applied to level design, with regard to both map analysis and placement of game elements, and we found various graphs and metrics that allow to extract interesting information from a map starting from its All-Black representation. Then, we designed an approach that allows to employ this information to place game element in a meaningful way, considering the different requirements of each object, that we defined by analyzing how level design influences the up-player vs down-player dynamic.

\par

Finally, we tested our framework by performing an experiment to analyze how the placement of spawn points influences the up-player vs down-player dynamic. With this experiment we were able to validate the placement methods that we have defined and we managed to observe how the map layout influences the disposition of game elements. In this way, we proved our graph-based approach to be useful both for map analysis and for the contextual positioning of game elements.

% ISSUES %

\section{Known issues and possible criticism}

The main issue with the framework is that it does not have neither artificial agents nor the support for online multiplayer and this limits its possible applications.

\par

For what concerns graph analysis, the rules that we have defined for placing game elements could be criticized for a lack of a strong theoretical basis, since as we have seen there is still no common ground for what concerns level design. Moreover, we assigned the weights used in the placement heuristics empirically, making various attempts and choosing the weights that produced the disposition of game elements most coherent with the rules we defined. Despite this, the experiment proved both the rules and the weight assignment to be effective.

% FUTURE %

\section{Future developments}

Two major features that should be implemented in the framework are an artificial intelligence for agents and the support for online multiplayer, since they would allow to significantly increase the possible applications of our work. Moreover, to make the framework more complete and allow to directly generate well designed maps, it would be a great improvement to implement the map analysis and the game element placement directly in the framework, instead of performing them using an external tool.

\par

In chapter \ref{ss:interesting_metrics} we listed many metrics that can give interesting information about the layout of a map, but we have used only some of them to define the placement heuristics. An interesting development would be to include more of them, in particular the ones that allow to define areas of the map, like \<Periphery> and \<Center>. As we have highlighted, weapons require a specific treatment when positioned, since their overall damage, strengths and weakness should influence their place in the map, and such metrics can be employed to define the areas that better suit each weapon. These new heuristics, as well as the already defined ones, would benefit of an experimental analysis similar to the one used to validate the heuristics for the placement of spawn points. Another improvement would be the extension of the analysis performed via graph to multi-level maps. Moreover, as we have already highlighted in the thesis, the graphs we have defined could be used for the individuation and analysis of design patterns.

\par

Finally, it would be interesting to design an evolutionary process that generates maps and places resources using a fitness function addressed to the up-player vs down-player dynamic.

% CHAPTER 7 - CONCLUSIONS %

\include{chapters/chapter7}

% BACK MATTER %

\backmatter

% BIBLIOGRAPHY %

\bibliography{bibliography} 
\bibliographystyle{ieeetr}

\end{document}