\chapter{The graph-based tool}

% INTRODUCTION %

In this chapter we describe the tool that we have developed to perform analysis and populating of pre-generated maps using \<Graph Theory>. After a quick overview, we introduce the analysis capabilities of this tool and then we present how we have employed them, together with Graph Theory, to strategically place resources (including spawn points) in pre-generated maps.

% DESCRIPTION %

\section{Description of the tool}

This tool generates different kind of graphs, starting from the text and the All-Black representation of a map, that are used to perform various analysis and manipulation operations.

\par

The use of All-Black format is convenient, because it provides by default a logical division of the map in different areas and it allows our tool to be applied by other researchers, since as we have seen the All-Black format is widely used in this field. We have used this tool to position objects in a pre-generated map, but, for instance, it could be used to address the identification and definition of design patterns from an unfamiliar perspective or for \<direct evaluation> in Search Based PCG.

\par 

We developed this tool in \<Python>, since it offers many solid Graph Theory libraries, as \<NetworkX>, that is the one we used.

% ANALYSIS %

\section{Analysis of pre-generated maps}

The analysis is performed by generating different kind of undirected graphs, each one used to highlight a different feature of the map in question.

\subsection{Outlines graph}

The \<outlines graph> is generated starting from the All-Black representation of a map and is obtained by associating a node to every vertex of every room and corridor and by connecting the non-adjacent ones that belong to the same outline. This graph has a single kind of node (\<vertex node>) that contains the coordinates of the tile it represents, which are used to position the node when the graph is visualized. Figure \ref{img:graph_out} shows an example of this graph.

\par

This graph can be used to visualize the rooms which compose the map.

\subsection{Reachability graphs}

Our tool can generate various kinds of \<reachability graphs> that represent various ways in which an entity can navigate a map. In these graphs a node represents a position that an entity can reach, whereas an edge indicates a viable path from a position to another.

\subsubsection{Tiles graph}

The \<tiles graph> is generated starting from the text representation of a map and is obtained by associating a node to each empty tile and by connecting each node to its corresponding 8-neighbors. The horizontal and vertical edges have cost $1$, whereas the diagonal ones have cost $\sqrt{2}$. This graph has a single kind of node (\<tile node>) that contains the coordinates of the tile it represents, which are used to position the node when the graph is visualized. Figure \ref{img:graph_tile} shows an example of this graph.

\par

This graph can be used to find the minimum distance that separates two cells, along with the shortest path that connects them.

\subsubsection{Rooms graph}

The \<rooms graph> is generated starting from the All-Black representation of a map and is obtained by associating a node to each room and corridor and by connecting nodes which corresponding rooms or corridors overlap, using as weight the Euclidean distance of their central tile. This graph has a single kind of node (\<room node>) used to represent both rooms and corridors that contains the coordinates of the closest and furthest vertex of the room from the origin. When visualized, each node is positioned on the coordinates of the central tile of the room it represents. Figure \ref{img:graph_room} shows an example of this graph.

\par

This graph can be used to analyze the topology of a map, in order to find loops, choke points, central areas and other kind of structures.

\subsubsection{Rooms and resources graph}

The \<rooms and resources graph> is an extension of the room graph, which also include resources as nodes, that are connected to the nodes corresponding to the rooms and corridors which contain them. In addition to the room node inherited form the rooms graph, this graph has a node to represent resources (\<resource node>) that contains the coordinates of the resource, which are used to visualize the node, and the character associated to the resource. Figure \ref{img:graph_room_res} shows an example of this graph.

\subsection{Visibility graph}

The \<visibility graph> is generated starting from the text representation of a map and is obtained by associating a node to each empty tile and by connecting each node to all the tiles that are visible from that node. For two tiles to be respectively visible, it must be possible to connect them with a line without crossing any filled tile. This graph has a single kind of node (\<visibility node>) that contains the coordinates of the tile it represents, which are used to position the node when the graph is visualized, and its \<degree centrality>, i.e. the number of edges incident to that node. 

\par

To make this graph easier to read by the user, the tool associates a color to the nodes, which ranges from blue, for the ones with the minimum visibility, to red, for the ones with the maximum visibility. This can be seen in figure \ref{img:graph_visibility}.

\par

This graph can be used to analyze which areas of the map are more exposed and which ones are more repaired.

\begin{figure}[]
	\centering
  	\begin{subfigure}[t]{0.45\linewidth}
		\includegraphics[width=\linewidth]{graph_divisive}
     		\caption{The map.}
     		\label{img:graph_divisive}
 	\end{subfigure}
  	\begin{subfigure}[t]{0.45\linewidth}
    		\includegraphics[width=\linewidth]{graph_out}
    		\caption{The outlines graph of the map.}
     		\label{img:graph_out}
  	\end{subfigure}
  	\begin{subfigure}[t]{0.45\linewidth}
    		\includegraphics[width=\linewidth]{graph_tile}
    		\caption{The tiles graph of the map.}
     		\label{img:graph_tile}
  	\end{subfigure}
  	\begin{subfigure}[t]{0.45\linewidth}
    		\includegraphics[width=\linewidth]{graph_room}
    		\caption{The rooms graph of the map.}
     		\label{img:graph_room}
 	\end{subfigure}
  	\begin{subfigure}[t]{0.45\linewidth}
    		\includegraphics[width=\linewidth]{graph_room_res}
    		\caption{The rooms and resources graph of the map.}
     		\label{img:graph_room_res}
  	\end{subfigure}
  	\begin{subfigure}[t]{0.45\linewidth}
    		\includegraphics[width=\linewidth]{graph_visibility}
    		\caption{The visibility graph of the map.}
     		\label{img:graph_visibility}
  	\end{subfigure}	
	\caption{A map and all the graphs that the tool can generate from it.}
\end{figure}

\subsection{Interesting metrics}

Considering the graphs, in particular the ones with room nodes, the following metrics defined by Graph Theory provide interesting information about the layout of a map:

\begin{itemize}
\item \<Degree centrality>: defined for a node, it is the number of edges that the node has. If the node represents a room, it measures how many entrance or exits the room has. 
\item \<Closeness centrality>: defined for a node, it measures its centrality in the graph, computed as the sum of the lengths of the shortest paths between the node and all other nodes in the graph. If the node represents a room, it measures how central the room is.
\item \<Betweenness centrality>: defined for a node, it measures its centrality in the graph, computed as the number of shortest paths connecting the nodes that pass through the node. If the node represents a room, it measures how central the room is.
\item \<Connectivity>: defined for a graph, it is the minimum number of elements (nodes or edges) that need to be removed to disconnect the remaining nodes from each other. If the graph represents a map, it measures the existence of isolated areas.
\item \<Eccentricity>: defined for a node, it is the maximum distance from the node to all other nodes in the graph. If the node represents a room, it measured how isolated the room is.
\item \<Diameter>: defined for a graph, it is the maximum eccentricity of its nodes. If the graph represents a map, it measures the size of the map.
\item \<Radius>: defined for a graph, it is the minimum eccentricity of its nodes. If the graph represents a map, it measures how distanced the rooms are from each other.
\item \<Periphery>: defined for a graph, it is the set of nodes with eccentricity equal to the diameter. If the graph represents a map, it defines its peripheral areas.
\item \<Center>: defined for a graph, it is the set of nodes with eccentricity equal to the radius. If the graph represents a map, it defines its central areas.
\item \<Density>: defined for a graph, it ranges from 0 to 1, going from a graph without edges to a complete graph. If the graph represents a map, it measures how complex it is.
\end{itemize}

% POPULATING %

\section{Populating of pre-generated maps}

We have defined multiple heuristics to populate a map with spawn points and other resources using the metrics that can be extracted from a graph. These heuristic are a mathematical transposition of rules and patterns concerning resource placement that we have extracted from the work of Tim Schäfer\cite{great1vs1}, who has performed an in depth analysis of multiplayer 1vs1 maps for \<Quake 2>\footnote{Id Software, 1997}.

\subsection{Rules and patterns for resources positioning}

The balance of a deathmatch game radically changes each time that a player is killed. If the game has more than two players, the player who won the fight does not gain any strategic advantage, since he still has the other players to face, whereas the defeated player is put at considerable disadvantage, because on death he loses all the weapons and ammunition that he has collected. In a 1vs1 match, a kill has an even stronger influence, since the surviving player has more weapons and ammunition and gains the complete control of the map, that comes with the chance of scoring another easy kill, as soon as the other player respawns, or of searching for additional equipment. Schäfer refers to the surviving player as \[up-player] and to the defeated one as \[down-player].

\par

To obtain a multiplayer map that is interesting and fun to play, it is important to consider the up-player vs down-player dynamic both when defining the map layout and when positioning resources.

\par

The spawn points, i.e. the locations where the down-player reappears, should be positioned in areas that are of low interest for the up-player and that are easy to leave. Obviously, central hubs and dead ends are a bad choice, whereas rooms with 2 or 3 exits are usually the best option. 

\par

For what concerns the resources, they must be placed considering both the up-player vs down-player dynamic and the characteristics of the resource itself. It is important to place the right amount of resources on the map, because too many would eliminate the need for exploration, whereas too few would disadvantage the down player. It is also important not to place too many powerful items in the same area or in boring spots, since the risk to obtain them should always be proportional to the strategical advantage they allow to achieve. It is important to consider that a powerful resource is interesting for both players, so it often acts as a \<point of collision>. The resources usually are of five kinds: health packs, \<armors>, \<power-ups>, ammunition and weapons. 

\par

The health packs are placed in zones that are safe or not too dangerous. They have no use for the down-player, that respawns with full health, but they can be useful for the up-player, if he has been damaged during the fight, whereas they always come in handy during a fight or after that one of the contenders disengages.

\par

Armors, which supply a second health that is consumed before the main one, are usually placed in spots that are aimed both at the down-player and at the up-player: objects that provide a small quantity of armor should be easy to achieve, whereas the ones that provide full armor should be placed in dangerous areas.

\par

Power-ups grant temporary advantages to the player who collects them, like invisibility or increased damage, and are placed in locations difficult to reach and contextual to their effect.

\par 

The position of a weapon and of its ammunition depends on the weapon itself. We can divide the weapons in three categories: \<weak>, \<medium> and \<strong>. Weak weapons are of a certain interest for the down-player, if he has not collected any other weapon yet, and of no interest for the up-player, so they are placed near spawn-points or in gaps where no other weapon is available, together with their ammunition. Medium weapons are of high interest for the down-player, since he needs to get one of them as soon as possible if he wants to face the up-player, so they are placed in areas that are easy to reach and the same goes for their ammunition. Finally, the strong weapons should be placed in areas that are strategically disadvantageous, like dead ends or vertically dominated areas, or difficult to reach. If a weapon is very contextual, i.e. it is useful in very few situations, it is usually placed in an area that allows to take advantage of its features, whereas a weapon that is strong in almost any situation is usually placed in an area where it cannot be used optimally (e.g. a rocket launcher in a small room).

\subsection{Placement process}
\label{ss:placement}

Starting from these considerations, we defined a process based on heuristics that allows to position any kind of resource. This process consists in two phases: the selection of a room and the selection of a tile inside the room.

\par

The selection of a room is performed considering two suitability criteria:

\begin{itemize}
\item \<Degree suitability>: defined by the function $D(r)$, where $r$ is a room node, it measures how much the degree centrality of the node matches the desired one.
\item \<Resources closeness suitability>: defined by the function $C_{res}(r)$, where $r$ is a room node, it measures how much the closeness of the node to the already placed resource nodes matches the desired one.
\end{itemize}

Given $G$ the rooms and corridors graph and $R \subset G$ the subset of room nodes, the room node which is selected is the one which maximizes the following weighted sum of functions:

\begin{align}
room = \argmax_{r \in R} (w_D  \times D(r) + w_{C_{res}}  \times C_{res}(r))
\end{align}

\noindent
Both functions should be defined to output a value in the range $[0,1]$.

\par

Once that the room is selected, the selection of a tile is performed considering three suitability criteria:

\begin{itemize}
\item \<Visibility suitability>: defined by the function $v(t)$, where $t$ is a tile node, it measures how much the visibility of the node matches the desired one.
\item \<Wall closeness suitability>: defined by the function $c_{wall}(t)$, where $t$ is a tile node, it measures how much the proximity of the corresponding tile to the walls matches the desired one.
\item \<Resources closeness suitability>: defined by the function $c_{res}(t)$, where $t$ is a tile node, it measures how much the proximity of the node with already placed resource nodes matches the desired one.
\end{itemize}

Given $G$ the tiles graph and $T \subset G$ the subset of tiles contained by the selected room, the tile which is selected is the one which maximizes the following weighted sum of functions:

\begin{align}
room = \argmax_{t \in T} (w_v \times v(t) + w_{c_{wall}}  \times c_{wall}(t) + w_{c_{res}}  \times c_{res}(t))
\end{align}

\noindent
All three functions should be defined to output a value in the range $[0,1]$.

\par

The process can be repeated as many time as needed, after having updated the graphs with the newly added resource. To obtain the best result, the resources should be placed in the order their heuristics are presented.

\subsection{Spawn points placement}

We have defined three heuristics for the placement of spawn points, the first one follows the process defined in subsection \ref{ss:placement}, the other two do not.

\subsubsection{Low risk heuristic}

This method selects rooms that have a small number of connections, are not dead ends and are distant from each other and for each one of them places the spawn point on the tile that offers the best balance between low visibility and distance from the walls. In this way spawn points are placed in passageways that are sheltered and easy to leave. 

\par

Given $G$ the rooms and corridors graph, $R \subset G$ the subset of room nodes, $S \subset G$ the subset of resource nodes (composed exclusively by the already positioned spawn points), the most suitable room for containing a spawn point is selected using

\begin{align}
\label{eq:lowriskdeg}
D(r) = \begin{cases}
    		\hfil 0 & \text{if } \degcent((r)) = 1 \\
    		1 - \cfrac{\degcent(r) - \min_{r' \in R}\degcent(r')}{\max_{r' \in R}\degcent(r') - \min_{r' \in R}\degcent(r')} & \text{if } \degcent(r) \neq 1 \
  	\end{cases}
\end{align}

\begin{align}
\label{eq:lowriskres}
C_{res}(r) = \min_{n \in G}
  	\begin{cases}
    		\hfil 1 & \text{if } n \notin S \\
    		\cfrac{\spl(r, n)}{\diam(G)} & \text{if } n \in S \
  	\end{cases}
\end{align}

\noindent
where $deg$ denotes the connectivity degree of a node. Equation \ref{eq:lowriskdeg} promotes rooms with few passages but that are not dead ends, whereas equation \ref{eq:lowriskres} promotes rooms that are distant from the already placed spawn points. Both are normalized in the range $[0,1]$. We experimentally found that the weights which provide the best results are $w_D = 1 $ and $ w_{C_{res}} = 0.5 $.

\par

Given G the visibility graph, $T \subset G$ the subset of tile nodes that belong to the room, $R \subset G$ the subset of tile nodes which contain a resource (composed exclusively by the already positioned spawn points), the most suitable tile for containing a spawn point is selected using

\begin{align}
\label{eq:lowvis}
v(t) = 1 - \cfrac{\degcent(t) - \min_{t' \in G}\degcent(t')}{\max_{t' \in G}\degcent(t') - \min_{t' \in G}\degcent(t')}
\end{align}

\begin{align}
\label{eq:lowwallclos}
c_{wall}(r) = \cfrac{\walld(t, room)}{ \maxwalld(room)}
\end{align}

\begin{align}
\label{eq:lowresclos}
 c_{res}(r) = \begin{cases}
    		\hfil 0 & \text{if } | S | = 0 \\
    		\min_{s \in S}\cfrac{\cartd(t, s)}{d}& \text{if }  | S | > 0 \
  	\end{cases} 
\end{align}

\noindent
where \<wall distance> is the distance of a tile from the walls of the room, computed as the sum of the minimum distances from the horizontal and vertical walls, and \<maximum wall distance> is the wall distance of the central tile of the room. Equation  \ref{eq:lowvis} promotes tiles with low visibility, equation \ref{eq:lowwallclos} promotes tiles that are distant from the walls, whereas equation \ref{eq:lowresclos} promotes tiles that are distant from the already placed spawn points. All three are normalized in the range $[0,1]$. We experimentally found that the weights which provide the best results are $ w_v = 1 $ , $ w_{c_{wall}} = 0.5 $ and $ w_{c_{res}}  = 0.5 $.

\subsubsection{Uniform heuristic}

This method selects rooms that are uniformly distributed in the map and for each one of them places the spawn point on the tile that offers the best balance between low visibility and distance from the walls. 

\par

The first room is selected at random, then, given $G$ the rooms and corridors graph, $R \subset G$ the subset of room nodes, $S \subset G$ the subset of resource nodes (composed exclusively by the already positioned spawn points), the remaining rooms are selected with the following heuristic:

\begin{align}
	room = \argmax_{r \in R} ( \min_{s \in S} (\spl(r, s) ))
\end{align}

Once that the room has been found, one of its tiles is selected using equations \ref{eq:lowvis}, \ref{eq:lowwallclos} and \ref{eq:lowresclos} with weights  $ w_{c_{wall}} = 0.5 $ and $ w_{c_{res}}  = 0.5 $.

\subsubsection{Random heuristic}

This method selects rooms at random and for each one of them places the spawn point on the tile that offers the best balance between low visibility and distance from the walls, using using equations \ref{eq:lowvis}, \ref{eq:lowwallclos} and\ref{eq:lowresclos} with weights  $ w_{c_{wall}} = 0.5 $ and $ w_{c_{res}}  = 0.5 $.

\subsection{Health packs placement}

Heath packs are placed in rooms that have a low-medium number of connections and are distant from each other on the tile that offers the best balance between medium visibility and distance from the walls. In this way the heath packs are placed in areas that are easy to reach and are not too exposed.

\par

Given $G$ the rooms and corridors graph, $R \subset G$ the subset of room nodes, $S \subset G$ the subset of resource nodes (composed by the already positioned spawn points and health packs), the most suitable room for containing a health pack is selected using

\begin{align}
\label{eq:lowriskdegh}
D(r) = \cfrac{f_{deg}(r) - \min_{r' \in R}f_{deg}(r')}{\max_{r' \in R}f_{deg}(r') - \min_{r' \in R}f_{deg}(r')} 
\end{align}

\begin{align}
\label{eq:lowriskresh}
C_{res}(r) = \min_{n \in G}
  	\begin{cases}
    		\hfil 1 & \text{if } n \notin S \\
    		\cfrac{\spl(r, n)}{\diam(G)} & \text{if } n \in S \
  	\end{cases}
\end{align}

\noindent
with

\begin{align}
\label{eq:degfit}
f_{deg} = \intd(\degcent(r),0.3,0.5)
\end{align}

\begin{align}
\label{eq:intervaldistance}
\intd(v, v_{min}, v_{max}) = | ( |v_{min}| - |v | ) | + | (|v_{max}| - |v | )| 
\end{align}

\noindent
where \<interval distance> is the distance of a value form the extremes of an interval and it is used to estimate how a value fits an interval. Equation \ref{eq:lowriskdegh} promotes rooms with a low-medium number of passages, whereas equation \ref{eq:lowriskresh} promotes rooms that are distant from the already placed resources. Both are normalized in the range $[0,1]$. We experimentally found that the weights which provide the best results are $w_D = 1 $ and $ w_{C_{res}} = 0.5 $.

\par

Given G the visibility graph, $T \subset G$ the subset of tile nodes that belong to the room, $R \subset G$ the subset of tile nodes which contain a resource, the most suitable tile for containing a health pack is selected using

\begin{align}
\label{eq:lowvish}
v(t) = 1 - \biggg| 0.5 - \cfrac{\degcent(t) - \min_{t' \in G}\degcent(t')}{\max_{t' \in G}\degcent(t') - \min_{t' \in G}\degcent(t')} \biggg| 
\end{align}

\noindent
that promotes tiles with medium visibility. Equations \ref{eq:lowwallclos} and \ref{eq:lowresclos} are used for $c_{wall}(r)$ and $c_{res}(r)$ , respectively. All three are normalized in the range $[0,1]$. We experimentally found that the weights which provide the best results are $ w_v = 1 $, $ w_{c_{wall}} = 0.25 $ and $ w_{c_{res}}  = 0.5 $.

\subsection{Ammunition placement}

Ammunition is placed in rooms that have either a low-medium or a high number of connections and are distant from each other on the tile that offers the best balance between high visibility and distance from the walls. In this way ammunition is placed in areas that are either easy or difficult to reach and easy to spot.

\par

Given $G$ the rooms and corridors graph, $R \subset G$ the subset of room nodes, $S \subset G$ the subset of resource nodes (composed by the already positioned spawn points, health packs and ammunition), the most suitable room for containing ammunition is selected using equations \ref{eq:lowriskresh} and \ref{eq:lowriskdegh}, with

\begin{align}
\label{eq:degfita1}
f_{deg} = \intd(\degcent(r),0.2,0.4)
\end{align}

\noindent
for obtaining rooms with a low-medium number of connections and
 
\begin{align}
\label{eq:degfita2}
f_{deg} = \intd(\degcent(r),0.8,0.9)
\end{align}

\noindent
for obtaining rooms with a high number of connections. We experimentally found that the weights which provide the best results are $w_D = 1 $ and $ w_{C_{res}} = 0.25 $.

\par

Given G the visibility graph, $T \subset G$ the subset of tile nodes that belong to the room, $R \subset G$ the subset of tile nodes which contain a resource, the most suitable tile for containing ammunition is selected using

\begin{align}
\label{eq:highvisa}
v(t) = \cfrac{\degcent(r) - \min_{r' \in R}\degcent(r')}{\max_{r' \in R}\degcent(r') - \min_{r' \in R}\degcent(r')}
\end{align}

\noindent
that promotes tiles with high visibility. Equations \ref{eq:lowwallclos} and \ref{eq:lowresclos} are used for $c_{wall}(r)$ and $c_{res}(r)$, respectively. All three are normalized in the range $[0,1]$. We experimentally found that the weights which provide the best results are $ w_v = 1 $, $ w_{c_{wall}} = 0.25 $ and $ w_{c_{res}}  = 0.5 $.

\subsection{Weapon placement}

For what concerns the weapons provided by the framework we have not defined any specific function yet, but, beside the assault rifle which is always available to the player, they could be placed as follows:

\begin{itemize}
\item \<Shotgun>: since it is a medium damage weapon it could be positioned in a room that has a medium number of connections and is relatively close to a spawn point. 
\item \<Rocket launcher>: since it is a high damage weapon it could be positioned in a room that has a high number of connections, creating an interesting collision point, or in a dead end, where its utility is limited.
\item \<Sniper rifle>: since it is a high damage weapon it could be positioned in a room that has a high number of connections, creating an interesting collision point.
\end{itemize}

% SUMMARY %

\section{Summary}

In this chapter we analyzed the tool that we have developed to perform map analysis and populating using Graph Theory. After an overview of the graphs and metrics that allow to highlight interesting information about a map, we listed the rules and patterns commonly used to position resources in deathmatch maps and we described how we converted them into heuristics.