\chapter{Experiment on spawn points placement heuristics}

% INTRODUCTION %

In this chapter we describe the experiment that we have performed for validating the placement heuristics for spawn points described in the previous chapter and, at the same time, for testing the data-collection capabilities of our framework, that was used to setup and manage the experiment.

% DESCRIPTION %

\section{Description}

With this experiment we analyzed how the placement of spawn points influences the up-player vs down-player dynamic. The experiment tries to recreate the situation where the up-player, once killed his opponent, tries to find him as soon as possible, just after his respawn, to score another easy kill. As we have seen, a well-designed map should slow down this operation by having its spawn points in areas that are not central, are easy to leave and covered. The \<low risk> heuristic described in chapter \ref{sss:lowrisk} has been designed to place the spawn points according to such criteria, so, to prove its effectiveness, we decided to confront how this facet of the up-player vs down-player dynamic changes with respect to spawn points placed with the \<uniform> heuristic, defined as well in chapter \ref{sss:uniform}, which selects the areas to contain spawn to be distributed evenly inside the map.

\par

To highlight this particular dynamic, we designed a game mode where the user, which represents the up-player, must find and destroy a static target, which represents the down-player, as many times as possible before times runs out. Each time that the user destroys a target, it respawns at a random spawn point. The user cannot die and has infinite ammunition, so he does not have to look for resources.

% SETUP %

\section{Setup}

For this experiment, we setup the \<Experiment Manager> to propose in each play session a quick tutorial, two matches and a survey. The experiment was composed by three \<studies>, corresponding to different pre-generated maps, each one composed by two \<cases>, one that corresponded to the \<low risk> spawn points distribution and one that corresponded to a pool of five \<uniform> spawn point distributions. In a play session, the user played the same map twice, once with the low risk distribution and once with one of the uniform distributions, in a random order and with the map flipped in one of the two matches. We used the survey to profile how much the user was familiar with video games and FPS, to evaluate his skill and to get a feedback about in which of the two maps it was harder to find the targets. The experiment was deployed online and played by the users via browser on their own computer.

\par

Each match had \<Target Hunt> as game mode. The duration of the match was set to three minutes, the list of spawnable entities consisted of just one target and the only weapon available to the player was the \<assault rifle> with infinite ammunition. The pre-generated maps were stored as text files and were loaded by the \<Divisive Generator> and displayed with the \<Prefab Assembler>. 

\par
 
For each match, a complete game log was saved, along with the following performance metrics, saved in a separate log:
 
 \begin{itemize}
\item \<TargetLogs>: this field contains a list of all the targets that the user managed to destroyed. Each entry contains a timestamp of when the target was destroyed, the coordinates of the target, the coordinates of the user, the distance covered by the user and the time passed during the lifespan of the target.
\item \<Shots>: the total number of projectiles shot by the user.
\item \<Hits>: the number of projectiles that hit a target.
\item \<Accuracy>: the percentage of projectiles that hit a target.
\item \<Kills>: the total number of targets destroyed by the user.
\item \<Distance>: the total distance covered by the user during the match, considering cells of unitary width.
\item \<AvgKillTime>: the duration of the match divided by the number of kills.
\item \<AvgKillDistance>: the total distance divided by the number of kills.
\end{itemize}

\noindent
The answers to the survey were saved as well.
 
 \par
 
The performance of the player is measured by \<AvgKillTime>, that is also an indicator of how difficult it is to find targets in the map.

\par
  
We have selected three procedurally generated maps that present radically different layouts and we populated each one of them using both the low risk and the uniform heuristic:

\begin{itemize}
\item \<Arena>: this map presents a wide open arena, two sides of which are adjacent to parallel corridors with many openings. As the visibility heatmap in figure \ref{img:arena_visibility} shows, the central arena allows to control most of the map, whereas the corridors offer some repair and perfect spots to place spawn points. Figure \ref{img:arena_safe} shows the spawn points positioned using the low risk heuristic, whereas figure \ref{img:arena_uniform} shows one of the five configuration produced using the uniform heuristic.
\item \<Corridors>: this map presents many small rooms connected by long corridors. As it can be seen in figure \ref{img:corridors_visibility}, there is no area that allows to control the others and the only point with high visibility are the ones where corridors intersect. Figure \ref{img:corridors_safe} shows the spawn points positioned using the low risk heuristic, whereas figure \ref{img:corridors_uniform} shows one of the five configuration produced using the uniform heuristic.
\item \<Intense>: compared to the previous two, this map presents an intermediate layout, since it has both open areas and small rooms connected by corridors. As it can be seen in figure \ref{img:intense_visibility}, this reflects also on the visibility, that is high in the central open arenas and low in the remaining sections of the map. Figure \ref{img:intense_safe} shows the spawn points positioned using the low risk heuristic, whereas figure \ref{img:intense_uniform} shows one of the five configuration produced using the uniform heuristic.
\end{itemize}

\par

It is important to observe that the low risk and the uniform heuristics select rooms to place a spawn point with two different criteria, but they employ the same logic when selecting a tile inside a room. This means that the two heuristics select tiles which have similar visibility conditions, so the player's performance difference depends exclusively on how the rooms have been selected. This is observable in figures \ref{img:arena}, \ref{img:corridors} and \ref{img:intense}: when the uniform heuristics happens to select a room that has been selected also by the low risk one, the spawn point is placed exactly on the same tile.

\begin{figure}[tp]
	\centering
	\begin{subfigure}[t]{0.3\linewidth}
    		\includegraphics[width=\linewidth]{arena_visibility}
     		\caption{Heatmap showing the visibility of the level.}
		\label{img:arena_visibility}
  	\end{subfigure}  	
  	\hfil
  	\begin{subfigure}[t]{0.3\linewidth}
    		\includegraphics[width=\linewidth]{arena_safe}
     		\caption{Spawn points (in red) placed using the safe heuristic.}
     		\label{img:arena_safe}
  	\end{subfigure}
  	\hfil
  	\begin{subfigure}[t]{0.3\linewidth}
    		\includegraphics[width=\linewidth]{arena_uniform}
     		\caption{Spawn points (in red) placed using the uniform heuristic.}
		\label{img:arena_uniform}
  	\end{subfigure}  
	\caption{``Arena'' map used in the experiment.}
	\label{img:arena}
\end{figure}

\begin{figure}[tp]
	\centering
  	\begin{subfigure}[t]{0.3\linewidth}
    		\includegraphics[width=\linewidth]{corridors_visibility}
     		\caption{Heatmap showing the visibility of the level.}
		\label{img:corridors_visibility}
  	\end{subfigure}  	  
  	\hfil
  	\begin{subfigure}[t]{0.3\linewidth}
    		\includegraphics[width=\linewidth]{corridors_safe}
     		\caption{Spawn points (in red) placed using the safe heuristic.}
     		\label{img:corridors_safe}
  	\end{subfigure}
  	\hfil
  	\begin{subfigure}[t]{0.3\linewidth}
    		\includegraphics[width=\linewidth]{corridors_uniform}
     		\caption{Spawn points (in red) placed using the uniform heuristic.}
		\label{img:corridors_uniform}
  	\end{subfigure}	
	\caption{``Corridors'' map used in the experiment.}
	\label{img:corridors}
\end{figure}

\begin{figure}[tp]
	\centering  	
  	\begin{subfigure}[t]{0.3\linewidth}
    		\includegraphics[width=\linewidth]{intense_visibility}
     		\caption{Heatmap showing the visibility of the level.}
		\label{img:intense_visibility}
  	\end{subfigure}  	
  	\hfil
  	\begin{subfigure}[t]{0.3\linewidth}
    		\includegraphics[width=\linewidth]{intense_safe}
     		\caption{Spawn points (in red) placed using the safe heuristic.}
     		\label{img:intense_safe}
  	\end{subfigure}
  	\hfil
  	\begin{subfigure}[t]{0.3\linewidth}
    		\includegraphics[width=\linewidth]{intense_uniform}
     		\caption{Spawn points (in red) placed using the uniform heuristic.}
		\label{img:intense_uniform}
  	\end{subfigure}
	\caption{``Intense'' map used in the experiment.}
	\label{img:intense}	
\end{figure}

% RESULTS %

\section{Results}

\begin{figure}
 	\centering
  	\begin{subfigure}[t]{0.8\linewidth}
    		\includegraphics[width=\linewidth]{bar_lowrisk}
     		\caption{Low risk heuristic.}
		\label{img:bar_lowrisk}
  	\end{subfigure}  	
  	\begin{subfigure}[t]{0.8\linewidth}
    		\includegraphics[width=\linewidth]{bar_uniform}
     		\caption{Uniform heuristic.}
     		\label{img:bar_uniform}
  	\end{subfigure}
	\caption[Distribution of the matches by number of kills for the two heuristics.]{Distribution of the matches by number of kills for the two heuristics. The matches are also classified by map.}
	\label{img:match_distribution}	
\end{figure}

The data collection campaign for this experiment produced a dataset composed by 27 samples, which distribution between the three maps can be seen in figure \ref{img:match_distribution}. Of 27 samples, 10 are pairs of matches played in map ``Arena'', 9 are pairs of matches played in map ``Corridors'' and 8 are pairs of matches played in map ``Intense''. 

\par

As we have seen, the performance of the player is measured by \<AvgKillTime> and by computing its mean value we observed that the users performed better in the matches associated to the uniform distribution with respect to the ones associated to the the low risk distribution. For the former the mean value of \<AvgKillTime> is $16.06$ seconds and the one of \<kills> is $11.88$ kills, whereas for the latter  the mean value of \<AvgKillTime> is $20.24$ seconds and the one of \<kills> is $10.06$ kills. The increase of 4 seconds on the average time needed to find a target confirms that spawn points placed with the low risk heuristic are more difficult to find and thus such heuristic is valid.

\par

To prove the statistic soundness of this result, we performed the \<Wilcoxon> statistical test\footnote{The \<Wilcoxon signed-rank test> is a \<non-parametric statistical hypothesis test> used to compare two matched samples to assess whether their population mean ranks differ.} by \<Pratt>\footnote{With respect to the standard Wilcoxon test, the one by Pratt considers also the observations for which the difference of the elements in the pair is zero. Since we have 4 cases where the number of kills is the same, we opted for this approach.}, using as matched samples the number of kills in maps with low risk heuristic and the number of kills in maps with uniform heuristic. The test was passed with $\alpha < 0.005$, one-tiled.

\begin{figure}[h]
 	\centering
  	\begin{subfigure}[t]{0.48\linewidth}
    		\includegraphics[width=\linewidth]{scatter_kills}
     		\caption{Kills.}
		\label{img:scatter_kills}
  	\end{subfigure}
  	\hfill
  	\begin{subfigure}[t]{0.48\linewidth}
    		\includegraphics[width=\linewidth]{scatter_avg}
     		\caption{AvgKillDistance.}
     		\label{img:scatter_avg}
  	\end{subfigure}
	\caption[Experiment outcomes by metrics.]{Experiment outcomes by metrics. The outcome associated to the low risk heuristics is on the horizontal axis, the one associated to the uniform heuristics is on the vertical axis.}
	\label{img:metrics}	
\end{figure}

The effect of the heuristics on the metrics that we have defined can be observed by plotting their values, assigning to the horizontal axis the value of the metric in maps populated with the low risk heuristic and to the vertical axis the value of the metric in maps populated with the uniform heuristic. Each point of such graph represents the outcome of a test whose coordinates are the value of the metric in the two matches. By tracing the bisector, it is easy to see for which of the two heuristics a metric is higher. If the points are scattered under the bisector, it means that the metric tends to be higher for the low risk heuristic, if they are scattered around the bisector it means that there is no significant difference, if they are scattered above the bisector it meas that the metric tends to be higher for the uniform heuristic. Figure \ref{img:scatter_kills} shows how the heuristics influence the number of kills: of 27 tests, 17 have more kills in the map  populated using the uniform heuristic, 6 have more kills in the map populated using the safe heuristic and 4 have the same number of kills. Figure \ref{img:scatter_distance} shows how the heuristics influence the average distance covered to find a target, which tends to be higher in maps whose spawn points have been placed using the low risk heuristics.

\par

The result of the uniform are more sèreaded also beacuse the maps are taken dorm a pool. If we analyze the single maps, we can see that with the low risk distribution the arena has mean 10.75, the corridors 10.44 and intense 8.75. The result ofa rena was expected, thanks to its central area that alows to easly control the rest of the map, and also the targets in corridosr to be harder to find than in arena, but would have expected to have intense in between the two, instead it relvealed to be the map where targets were harder to find, probably beacuse its tangled structure. If we consider the uniform heuristic we see an evident increase in arena, 11.88 that was contrary to our expecation, since we expected the situation to remain more or less the same, corridors, instead keeps the kill count almost the same 10.67, whereas intense has a dramatic increase to 12.75. A possible explenation is that corridors ha a really uniform structure, and once in a corridor you can controll all of that portion of the map, so there is no difference in were palcing the spawn point. Instead in arena, with the central area that dominates everything, there is a great difference if it is covered. The same is probably true with Intense, that sums to this the fact htat there are peripheal areal that slow down the serach if a spawn point is tactically palcd inside it.

\begin{figure}
 	\centering
	\includegraphics[width=0.8\linewidth]{percived}
	\caption{Comparison between the effective and the perceived difficulty.}
	\label{img:percived}		
\end{figure}

A final interesting observation concerns how the difficulty was perceived by the user, this cna be seen in figure. At the end of the experiment the user were asked to tell whihc of the two map they played was more difficult: if the first, the second or the were the same. We can see that generally the user percived as more difficult the map were they made more kill, or not perviced any difference at all. Sometimes, altohigh, palters rated as more difficult the map were they èerformed better.

\par

Finally, the comment of the player revealed that the intuition of flipping the map was a good one, since often they didn't recongnised that the two maps were the same.

% CONCLUSIONS %

\section{Conclusions}

