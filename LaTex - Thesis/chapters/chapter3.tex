\chapter{Features of the research framework}

% INTRODUCTION %

In this chapter we describe the \<framework> that we have developed to perform user-based online validation for researches in procedural content generation of multiplayer levels for Firsts Person Shooters. In the first section we give an overview of the framework, of its features and of its components, analyzing them one by one in the following sections.

% FRAMEWORK OVERVIEW 

\section{Framework overview}

We designed our framework with the objective of providing a valid alternative to the games currently employed as a validation tool in this research field. All the available options, like \<Cube 2: Sauerbraten>, are powerful tools to perform validation via artificial agents, but they are not suitable for user-based validation. A data-collection campaign based on these games requires to download the game or to take part in real-life play-test sessions, but these options discourage potential participants because they are significantly time-consuming. For this reason, we decided to develop a framework that is as light as possible, with a WebGL build weighting less than 10MB that can be played using any browser. The framework was developed with Unity.

\par

Since the purpose of this tool is to be used in research, we decided to support many map representation formats used in previous works and we designed our framework to be as modular, expansible and configurable as possible.

\subsection{The framework structure}

The framework collects data by assigning to the users \<matches> to play. A match is defined by the \<game mode> and by the \<map type>, which in turn is defined by the \<map topology> and by the \<map appearance>. The \<map topology> defines how the map is going to \<be> and depends on the algorithm used to generate it, whereas the \<map appearance> defines how the map is going to \<look> and depends on how the map is assembled. This implies that the map type defines a whole array of procedurally generated maps that share the same topology and appearance. Therefore, when referring to a match we are considering a specific game-mode played in a procedurally generated map. If needed, it is possible to use a pre-generated map instead of generating a new one, by providing it as input in one of the supported formats. In this case the \<map topology> defines how to interpret the input, that is then displayed considering the \<map appearance>.

\par

A match is defined by combining different modular \<Manager> objects, each of which controls a different aspect of the match. To assure their interchangeability, most modules are defined by their own abstract class.

\subsection{The Game Manager}

The \<Game Manager> is the module responsible for the overall behavior of a match. Each game mode consists in a different implementation of the Game Manager. It leans on the \<Map Manager> for the generation and the assembly of the map and on the \<Spawn Point Manager> for the spawn of entities. The Game Manager controls the life-cycle of the match, that can be divided in the following phases:

\begin{itemize}
\item \<Setup>, all the modules are initialized.
\item \<Generation>, the Map Manager generates and assembles the map.
\item \<Ready>, the Game Manager displays a countdown announcing the start of the game.
\item \<Play>, the Game Manager handles the game while the \<Experiment Manager> logs the actions of the player, if needed. This phase continues until an end condition is satisfied.
\item \<Score>, the Game Manager stops the game and displays the final score.
\end{itemize}

\subsection{The Spawn Point Manager}

The \<Spawn Point Manager> contains a list of all the spawn points displaced on the map, that is populated during the Generation phase by the Map Manager. When the Game Manager needs to spawn an entity, the Spawn Point Manager provides a random spawn point from the ones that have not been used in a certain amount of time. If no spawn point meets this condition, the extraction is made from the complete pool.

\subsection{The Map Manager}

\subsection{The Map Generator}

\subsection{The Map Assembler}

\subsection{The Object Displacer}

% MAP REPRESENTATION AND GENERATION %

\section{Map representation and generation}

% WEAPONS AND OBJECTS %

\section{Weapons and objects}

% GAME MODES %

\section{Game modes}

% LOGGING %

\section{Logging}

% EXPERIMENT MANAGEMENT %

\section{Experiment management}
