\chapter{\textit{Thanks}}

\textit{I would like to thank Assistant Professor Daniele Loiacono, that supported me during the long six months that led to the completion of this work.}

\par \mbox{}

\textit{Thank also to my whole family and to my colleague and friend Luca, essential companion in this journey.}

\par \mbox{}

\textit{Finally, I would like to thank Politecnico di Milano itself, of which I can call myself a proud student.}

\par \mbox{}

\textit{\rightline{Marco Ballabio}}

\chapter{Abstract}

\<Level design> plays a key role in the development of a video game, since it allows to transform the \<game design> in the actual \<gameplay> that the final user is going to experience. Nevertheless, we are still far from a scientific approach to the subject, with a complete lack of a shared terminology and almost no experimental validation for the most used techniques. Even if the video game industry doesn't acknowledge this problem, in the last years the academic environments have shown an increasing interest towards this subject. \\
We analyzed the main breakthroughs made in \<level design> research applied to the genre of \<First Person Shooters>, devoting particular attention to the ones that try to assist the design process by employing \<Search-Based Procedural Content Generation> combined with \<evolutionary algorithms>. We noticed that in most cases researchers recur to \<validation> via \<artificial agents>, since human play-test session are too time-consuming and allow to collect only a limited amount of data. Unfortunately, this solution decreases the scientific soundness of the obtained results, since the behavior of an AI, no matter how advanced, is different form the one of a human player. To solve this problem, we developed an \<open-source> \<framework> capable of deploying online experiments to collect data from real users via a browser FPS game. \\
We also explored a novel approach to procedural generation of contents, developing a tool that uses Graph Theory to displace spawn-points and resources in a procedurally generated map. Finally, we used our framework to validate this tool by means of an online data-collection campaign. 

\chapter{Sintesi}

Il \<Level design> gioca un ruolo chiave nello sviluppo di un videogioco, dal momento che permette di trasformare il \<game design> nell'effettiva esperienza di \<gameplay> che verrà sperimentata dall'utente finale. Nonostante ciò, siamo ancora lontani da un approccio scientifico verso la materia, a casusa della completa mancanza di un vocabolario condiviso e della quasi totale assenza di validazione sperimentale per le tecniche più comuni. Anche se l'industria tende ad ignorare questo problema, negli ultimi anni gli ambienti accademici hanno mostrato un crescente interesse verso questo campo. \\
Abbiamo analizzato le principali scoperte fatte nel campo del \<level design> applicato al genere dei \<First Person Shooter>, riservando particolare attenzione ai casi dove si usa la \<generazione procedurali di contenuti> tramite \<algoritmi evolutivi> per assistere il processo di design. Abbiamo notato che nella maggior parte dei casi i ricercatori ricorrono alla \<validazione> tramite \<agenti artificiali>, dal momento che organizzare sessioni di test con giocatori umani richiede molto tempo e permette di raccogliere una quantità limitata di dati. Dal momento che il comportamento di un'IA, per quanto avanzata, è molto diverso da quello di un giocatore umano, questa soluzione va a diminuire la solidità scientifica dei risultati ottenuti. Per ovviare a questo problema, abbiamo sviluppato un \<framework> \<open-source> per la raccolta di dati online da utenti reali, tramite un gioco FPS usufruibile da browser. \\
Abbiamo anche tentato un nuovo approccio alla generazione procedurale di contenuti, sviluppando uno strumento che utilizza la Teoria dei Grafi per disporre punti di respawn e risorse di vario tipo in una mappa generata proceduralmente. Abbiamo infine usato il nostro framework per validare questo strumento tramite una raccolta di dati online.