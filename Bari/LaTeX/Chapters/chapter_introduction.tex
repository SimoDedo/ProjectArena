\chapter{Introduction}

The development of video games is a complex process and the most relevant aspect, in addition to graphics and narrative, is undoubtedly \textit{gameplay}, which refers to the ways in which the player interacts with the game. The definition of a game's \textit{gameplay} is given by a set of rules, the \textit{game design}, which is presented to the user tangibly through the worlds in which the game takes place and the set of mechanics available to the player to interact with the world. The combination of these two elements, known as \textbf{level design}, plays a fundamental role in defining the overall gaming experience.

In this work, we will focus on analyzing the theory of \textit{level design} in \textbf{First-Person Shooter} (FPS) video games, one of the most popular and well-known video game genres, particularly in the context of competitive shooters. In such games, in addition to considering the greater immersion required by the game world due to the first-person view, it is necessary to pay particular attention to the careful balancing of gameplay mechanics and interactions between various play styles during matches in order to ensure enjoyable and interesting gaming experiences.

Despite the importance of level design in FPS games, the topic has been little explored in the scientific field, with a lack of specific standards and vocabulary. In many cases, the success of a level design depends solely on the designer's experience and personal techniques, without a true understanding of the reasons behind their functioning.

Only recently have academic environments begun to pay greater attention to the identification, definition, and understanding of various design patterns and techniques in order to develop new tools useful for level design. Among the analyzed techniques, the use of evolutionary algorithms for procedural map generation seems particularly interesting, considering their effectiveness in other fields of application.

However, the use of evolutionary algorithms requires the availability of game data to guide the search for such procedures. To obtain this data, it is possible to involve human players or use bots for match simulation. In this thesis, we have decided to integrate an artificial intelligence system for match simulation through bots to the open-source Framework developed by Ballabio in \cite{ballabio_framework}. 

We then leveraged these bots to test a new type of map encoding using evolutionary algorithm.

\section{Contents of the thesis}
This work is structured in the following way:

In the second chapter we provide an overview of the current state of the art for what concerns \textit{level design} and \textit{procedural content generation} with \textit{evolutionary algorithms} applied to FPS games. We then briefly describe balancing in video games and FPS.

In the third chapter we describe bots for video games, focusing on their purpose, their artificial intelligence model and some techniques to implement their decision making capabilities.

In the fourth chapter we describe the Ballabio Unity framework, starting with a brief overview of what it offers to then focus on the bot system we added for this thesis.

In the fifth chapter, after verifing how different bot profiles compare to the others, we leverage the Framework and the newly developed bots to test evolution of new maps using old and new map representations.

Finally, in the conclusion we sum up the work we have done, the results we obtained and the issues and limitations we faced.

Additionally, in Appendix A, we show the syntax used in this thesis to represent the behaviour trees used to model the decision making processes of our bots.

\section{Contribution of this thesis}
In this thesis, we have designed, implemented, and tested a new open-source bot system for first-person shooter games in Unity. Despite its rich set of features and tools, Unity does not provide any out-of-the-box solutions for bot development.

Using these bots, we have introduced a new genome for encoding maps, called \textit{Grid-graph}. \textit{Grid-graph} builds upon key concepts, such as \textit{Rooms} and \textit{Corridors}, used by its predecessors while addressing some of their limitations. We have then evaluated the use of \textit{Grid-graph} in evolving balanced 1v1 maps for games which feature an unbalance in the skill level of players, and compared its performance with the \textit{All-black} representation first introduced by Cardamone \cite{cardamone_evolving_maps}.