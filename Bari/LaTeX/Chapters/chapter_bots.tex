\chapter{Creating a bot}
\label{ch:chapter_two}%

The purpose of this chapter is to describe the bot introduced into the Unity framework to automate gameplay testing.

\section{Overview}

Our Unity bot is build up in several layers each which its own specific purpose. This design allows to streamline the whole decision process, which starts from the data \textit{sensed} by the bot and, after some processing steps, leads to the final sequence of actions that the bot takes and which alter the state of the world.
The layered architecture is pictured in figure \ref{fig:layered_bot_architecture} and is described more in details in the following sections.


\begin{figure}[h]
    \centering
    \includegraphics[width=1\textwidth]{Images/IOLayers.png}
    \caption{Layered architecture of the bot}
    \label{fig:layered_bot_architecture}
\end{figure}

\subsubsection{Sensing layer}

One of the simplest layers, the sensing layer purpose is to receive raw data from the world in order to allow components above it to process it. 
It is composed of two components:
\begin{itemize}
    \item \textbf{Damage Sensor}: Detects whether the entity has been damaged recently;
    \item \textbf{Sight Sensor}: Can be used to check if any specific item of entity is visible.
\end{itemize}

\subsubsection{Knowledge base layer}
This second layer has the task of storing the raw data received by the sensing layer or known in advance to allow further processing.
It is composed of:
\begin{itemize}
    \item \textbf{Target knowledge}: Stores the visibility and (if detected) position of a specific target up to a certain point in the past;
    \item \textbf{Pickup knowledge}: Stores the known status of pickups and TODO THIS SHOULD BE SEPARATED the expected respawn time (stores the latest time the pickup was known to be active.);
    \item \textbf{Map knowledge}: Stores the last time a specific \textbf{Area} of the map was visited.
    \item \textbf{Navigation system}: Used to calculate paths across the map.
\end{itemize}

\subsubsection{Decision layer}
The brain of the bot, this layer is responsible of choosing the best course of action that the bot should take every instant.
It is composed of many components, most of which serve as helper for the core of this layer: the \textbf{Goal Machine}, which will be described better later in this chapter.
This layer contains:
\begin{itemize}
    \item \textbf{Map navigation planner}: Can be used to select the best area in the map to visit;
    \item \textbf{Pickup planner}: Can be used to select the best pickup to collect;
    \item \textbf{Enemy detection tracker}: Can be used to 
\end{itemize}



\subsubsection{Forth layer}

The forth layer contains the actual brain of the bot and it's the one responsible of making decisions which influence the state of the bot inside the world (e.g. fetch medkit, flee, chase enemy, explore map, ...)

The layers are organized so that each one depends only on those underneath it.



How to introduce the 
